\chapter{Overview}\label{sec:overview}

\section{Workflow Overview}

\cxoneflow has origins in CxFlow for the CxSAST product (CxSAST is the predecessor to \cxonens).  CxFlow
had a variety of functions and deployment options related to orchestrating scans in CxSAST and sending
result feedback to issue trackers.  \cxoneflow will also orchestrate scans but is adapted for the
concepts of \cxonens.

The \cxoneflow logic for how scans are orchestrated is very similar to that of CxFlow.  The basic
logic flow is that scans are executed when:

\begin{itemize}
    \item If a Push is made to a repository's protected branch, that protected branch is scanned.
    \item If a Pull Request is opened that targets a protected branch, a scan is performed on
    the source branch.
    \item If a Push is made to a branch that is the source of an open Pull Request that targets
    a protected branch, the open Pull Request branch is scanned.
\end{itemize}


\cxoneflow follows this workflow logic upon the receipt of a webhook event payload generated by the
source control management system (SCM). The code from the repository to be scanned is cloned by \cxoneflowns, 
collected into a zip file, then submitted for a scan to \cxonens.  When the scan is submitted,
the locally cloned code is deleted.


\section{Deployment Overview}

The method of deployment for \cxoneflow is intended to integrate scanning of all enterprise repositories
with a minimal amount of configuration.  The best method for deployment is to configure source control web
hooks where they will emit events for the largest possible number of repositories.  In many source control
systems, this can be done at a global or organization level.  The web hooks will be configured to send events
to a \cxoneflow endpoint specific to the type of source control system.

Figure \ref{fig:cxoneflow-deployment} is a \cxoneflow deployment diagram. The key points of the diagram:

\begin{itemize}
    \item A single instance or clustered install of \cxoneflow can be used as the endpoint for multiple
    SCM instances.
    \item There may be more than one instance of an SCM type.
    \item Each SCM may host logically separate organizations that can use the same \cxoneflow endpoint.
    The endpoint would be configured with multiple service instances to match multiple organizations.
    \item There may be multiple \cxone tenants where scans are to be invoked from an SCM or SCM
    logical group of repositories.
\end{itemize}

The \cxoneflow configuration allows each SCM receiver endpoint to be configured such that it orchestrates
scans in the correct target \cxone instance and tenant. \cxoneflow is compatible with \cxone hosted 
single-tenant, hosted multi-tenant, and self-hosted instances.

\begin{figure}[ht]
    \includegraphics[width=\textwidth]{graphics/cxoneflow-deployment.png}
    \caption{\cxoneflow Deployment Diagram}
    \label{fig:cxoneflow-deployment}
\end{figure}

\section{Webhook Deployment Topology}

The goal of any \cxoneflow deployment is to enable webhook event delivery for a
large number of code repositories without a significant amount of effort.  Each 
SCM has several webhook deployment topologies that can be utilized to allow
deployment with the most minimal amount of effort. External factors can influence
what would be considered the most minimal amount of effort.

There are many factors that should be considered when considering the deployment
approach.  In some cases, some of the SCM webhook deployment capabilities may
not be utilized based on other considered factors.  Some of the
other factors to consider can include:

\begin{itemize}
    \item Administrative control of one or more SCM instances.
    \item Administrative control of multiple organizations in an SCM instance.
    \item Existing SCM integrations.
    \item Repository usage patterns that align with multiple development teams and Software Development Lifecycles.
\end{itemize}

This section is intended to provide a brief overview of the supported SCM repository organization topologies.
This overview is not comprehensive; a deeper look into your organization's SCMs may be required
to understand the full scope of deployment work. Part \ref{part:scms} of this manual covers the
details for integrating \cxoneflow with each supported SCM type.

The tool \extlink{https://github.com/checkmarx-ts/cxone-flow-audit}{cxone-flow-audit} has been
developed to assist in automating and auditing large scale deployments in some scenarios.  As
of the writing of this manual, the capabilities are limited to Azure DevOps.  As time progresses,
more deployment assistance capabilities will be added.


\subsection{BitBucket Data Center Webhook Topology}

\chapter{BitBucket Data Center}

\section{About BitBucket Data Center}

BitBucket Data Center is an on-premise, self-hosted SCM that is often confused with BitBucket Cloud; while there are name and some
feature similarities in both products, the integration capabilities needed by \cxoneflow
are significantly different.  This section covers configuring \cxoneflow for use with BitBucket Data Center.

The \cxoneflow endpoint \texttt{/bbdc} is the handler for all webhook event
payloads originating from BitBucket Data Center.  

\section{Webhook Topology}

\chapter{BitBucket Data Center}

\section{About BitBucket Data Center}

BitBucket Data Center is an on-premise, self-hosted SCM that is often confused with BitBucket Cloud; while there are name and some
feature similarities in both products, the integration capabilities needed by \cxoneflow
are significantly different.  This section covers configuring \cxoneflow for use with BitBucket Data Center.

The \cxoneflow endpoint \texttt{/bbdc} is the handler for all webhook event
payloads originating from BitBucket Data Center.  

\section{Webhook Topology}

\chapter{BitBucket Data Center}

\section{About BitBucket Data Center}

BitBucket Data Center is an on-premise, self-hosted SCM that is often confused with BitBucket Cloud; while there are name and some
feature similarities in both products, the integration capabilities needed by \cxoneflow
are significantly different.  This section covers configuring \cxoneflow for use with BitBucket Data Center.

The \cxoneflow endpoint \texttt{/bbdc} is the handler for all webhook event
payloads originating from BitBucket Data Center.  

\section{Webhook Topology}

\input{scms/topology/bbdc.tex}

\section{Webhook Configuration}

Figure \ref{fig:bbdc-project-config} shows the BitBucket Data Center project configuration screen.  The
project key will appear in clone URLs and can be used as part of the regular expression 
placed in the \texttt{repo-match} configuration element.  Please see Section \ref{sec:yaml-config} 
for a description of the \texttt{repo-match} configuration element. The project's "Webhooks" configuration
can be used to configure the \cxoneflow endpoint \texttt{/bbdc} to receive webhook events for each repository in the organization.  

\begin{figure}[ht]
    \includegraphics[width=\textwidth]{graphics/bbdc-project-config.png}
    \caption{BitBucket Data Center Project Configuration}
    \label{fig:bbdc-project-config}
\end{figure}

\begin{figure}[ht]
    \includegraphics[width=\textwidth]{graphics/bbdc-webhook-config.png}
    \caption{BitBucket Data Center Webhook Configuration}
    \label{fig:bbdc-webhook-config}
\end{figure}


Figure \ref{fig:bbdc-webhook-config} shows a typical webhook configuration.  The \textit{Secret} is how \cxoneflow
validates the origin of the event payload.  The configuration element \texttt{shared-secret}, as described
in Section \ref{sec:yaml-config}, should be configured with the webhook secret value.  If \cxoneflow
is running at the specified URL endpoint, the "Test Connection" button will send a diagnostic ping
and receive back a positive response.  If the connection test fails, please ensure that \cxoneflow is running
at the address specified in the URL field and that the BitBucket Data Center server can make a connection
to that URL.

If the webhook is configured at the Project scope, the events sent apply to all repositories contained
within the project.  Figure \ref{fig:bbdc-repo-event-config} shows the configured repository-level webhook 
events that will send a webhook payload to the \cxoneflow endpoint. 
Figure \ref{fig:bbdc-pr-event-config} shows the configured pull-request events that will be sent to 
the \cxoneflow endpoint.  The following events are currently supported:

\begin{itemize}
    \item Repository -> Push
    \item Pull Request -> Opened
    \item Pull Request -> Source branch updated
    \item Pull Request -> Modified
    \item Pull Request -> Approved
    \item Pull Request -> Changes requested
    \item Pull Request -> Declined
    \item Pull Request -> Unapproved
    \item Pull Request -> Merged
    \item Pull Request -> Deleted
\end{itemize}


\begin{figure}[ht]
    \includegraphics[width=\textwidth]{graphics/bbdc-repository-event-config.png}
    \caption{BitBucket Data Center Webhook Repository Event Config}
    \label{fig:bbdc-repo-event-config}
\end{figure}

\begin{figure}[ht]
    \includegraphics[width=\textwidth]{graphics/bbdc-pr-event-config.png}
    \caption{BitBucket Data Center Webhook Pull Request Event Config}
    \label{fig:bbdc-pr-event-config}
\end{figure}


\section{\cxoneflowtext\space HTTP Access Tokens}

Current versions of BitBucket Data Center now reject the use of Basic Authorization using
a username and password. While it may be possible to use Basic Authorization to access the SCM, typically
this is a configuration that should be avoided.  In addition to Basic Authorization breaking upon BitBucket Data Center
server upgrades, an interactive user account may be subject to password changes and Captcha verification that
can break \cxoneflow operations.  

It is generally best to utilize a user HTTP access token for SCM connection
configurations \texttt{api-auth}.  If a user HTTP access token is used, the user can be granted
permissions for multiple projects; this will make the \cxoneflow configuration much easier. 

BitBucket Data Center can create project HTTP access tokens that are similar in use to
a user HTTP access token.  A project HTTP access token is not tied to a specific user, but the
access scope is limited to accessing repositories in the project where it was generated.
If using project HTTP access tokens, one project HTTP access token is required for \cxoneflow
to handle events emitted from repositories in each corresponding project. This also implies that
the \cxoneflow configuration will require one SCM service configuration per project with an
associated \texttt{repo-match} regular expression so the correct project token is used when
orchestrating the scan.

Please refer to Section \ref{sec:yaml-config} for more details about using the HTTP access token
to configure the BitBucket Data Center connection.

Figure \ref{fig:bbdc-token-config} shows the project "HTTP Access tokens" configuration.  The required
token permissions for \cxoneflow operations for any HTTP access token are:

\begin{itemize}
    \item Project read
    \item Repository read
    \item Repository write
\end{itemize}


\begin{figure}[ht]
    \includegraphics[width=\textwidth]{graphics/bbdc-token-config.png}
    \caption{BitBucket Data Center Project-Level HTTP Access Token Config}
    \label{fig:bbdc-token-config}
\end{figure}


User HTTP access tokens are tied to a user account but are not subject to password
changes, 2FA authentication, and captcha challenges with interactive logins.  A token generated
from a user as a service account can be granted the required permissions across multiple projects.
In this scenario, a single SCM service configuration would be able to handle events coming from
repositories in multiple projects.

\section{\cxoneflowtext\space SSH Keys}

While performing scan orchestration, \cxoneflow does access the BitBucket Data Center API for
certain operations.  This requires a configuration in the \texttt{api-auth} configuration
element as described in Section \ref{sec:yaml-config}.  The \texttt{clone-auth},
described in Section \ref{sec:yaml-config}, is an optional element where the credentials
used for cloning code can be provided.  If \texttt{clone-auth} is not provided, cloning will
be attempted using the credentials defined by \texttt{api-auth}.

The \texttt{clone-auth} configuration can define an SSH private key for use in cloning.  This
will allow for a separate set of credentials or authentication methods between cloning and
API use.


\section{Protected Branches}

The \cxoneflow workflow, as described in Section \ref{sec:overview}, uses the concept of "Protected Branches"
to know when to invoke workflows.  BitBucket Data Center allows for the configuration of the branching model
at the project and repository level.  Some repositories inherit their branching model from the project
configuration, but the ability for this to be overridden at the repository level is an optional configuration.
The branching model applied to the repository that emitted the webhook event is used to determine which branches
are "Protected Branches" at the time \cxoneflow handles the event.

The project-level branching model configuration is shown in Figure \ref{fig:bbdc-branch-config}.  The
repository-level branching model configuration is similar in that both allow the definition of
"Development" and "Production" branches.  \cxoneflow considers any branch specified as a Development
or Production branch to be a "Protected Branch".

\begin{figure}[ht]
    \includegraphics[width=\textwidth]{graphics/bbdc-branch-config.png}
    \caption{BitBucket Data Center Project-Level Branch Config}
    \label{fig:bbdc-branch-config}
\end{figure}



\section{Webhook Configuration}

Figure \ref{fig:bbdc-project-config} shows the BitBucket Data Center project configuration screen.  The
project key will appear in clone URLs and can be used as part of the regular expression 
placed in the \texttt{repo-match} configuration element.  Please see Section \ref{sec:yaml-config} 
for a description of the \texttt{repo-match} configuration element. The project's "Webhooks" configuration
can be used to configure the \cxoneflow endpoint \texttt{/bbdc} to receive webhook events for each repository in the organization.  

\begin{figure}[ht]
    \includegraphics[width=\textwidth]{graphics/bbdc-project-config.png}
    \caption{BitBucket Data Center Project Configuration}
    \label{fig:bbdc-project-config}
\end{figure}

\begin{figure}[ht]
    \includegraphics[width=\textwidth]{graphics/bbdc-webhook-config.png}
    \caption{BitBucket Data Center Webhook Configuration}
    \label{fig:bbdc-webhook-config}
\end{figure}


Figure \ref{fig:bbdc-webhook-config} shows a typical webhook configuration.  The \textit{Secret} is how \cxoneflow
validates the origin of the event payload.  The configuration element \texttt{shared-secret}, as described
in Section \ref{sec:yaml-config}, should be configured with the webhook secret value.  If \cxoneflow
is running at the specified URL endpoint, the "Test Connection" button will send a diagnostic ping
and receive back a positive response.  If the connection test fails, please ensure that \cxoneflow is running
at the address specified in the URL field and that the BitBucket Data Center server can make a connection
to that URL.

If the webhook is configured at the Project scope, the events sent apply to all repositories contained
within the project.  Figure \ref{fig:bbdc-repo-event-config} shows the configured repository-level webhook 
events that will send a webhook payload to the \cxoneflow endpoint. 
Figure \ref{fig:bbdc-pr-event-config} shows the configured pull-request events that will be sent to 
the \cxoneflow endpoint.  The following events are currently supported:

\begin{itemize}
    \item Repository -> Push
    \item Pull Request -> Opened
    \item Pull Request -> Source branch updated
    \item Pull Request -> Modified
    \item Pull Request -> Approved
    \item Pull Request -> Changes requested
    \item Pull Request -> Declined
    \item Pull Request -> Unapproved
    \item Pull Request -> Merged
    \item Pull Request -> Deleted
\end{itemize}


\begin{figure}[ht]
    \includegraphics[width=\textwidth]{graphics/bbdc-repository-event-config.png}
    \caption{BitBucket Data Center Webhook Repository Event Config}
    \label{fig:bbdc-repo-event-config}
\end{figure}

\begin{figure}[ht]
    \includegraphics[width=\textwidth]{graphics/bbdc-pr-event-config.png}
    \caption{BitBucket Data Center Webhook Pull Request Event Config}
    \label{fig:bbdc-pr-event-config}
\end{figure}


\section{\cxoneflowtext\space HTTP Access Tokens}

Current versions of BitBucket Data Center now reject the use of Basic Authorization using
a username and password. While it may be possible to use Basic Authorization to access the SCM, typically
this is a configuration that should be avoided.  In addition to Basic Authorization breaking upon BitBucket Data Center
server upgrades, an interactive user account may be subject to password changes and Captcha verification that
can break \cxoneflow operations.  

It is generally best to utilize a user HTTP access token for SCM connection
configurations \texttt{api-auth}.  If a user HTTP access token is used, the user can be granted
permissions for multiple projects; this will make the \cxoneflow configuration much easier. 

BitBucket Data Center can create project HTTP access tokens that are similar in use to
a user HTTP access token.  A project HTTP access token is not tied to a specific user, but the
access scope is limited to accessing repositories in the project where it was generated.
If using project HTTP access tokens, one project HTTP access token is required for \cxoneflow
to handle events emitted from repositories in each corresponding project. This also implies that
the \cxoneflow configuration will require one SCM service configuration per project with an
associated \texttt{repo-match} regular expression so the correct project token is used when
orchestrating the scan.

Please refer to Section \ref{sec:yaml-config} for more details about using the HTTP access token
to configure the BitBucket Data Center connection.

Figure \ref{fig:bbdc-token-config} shows the project "HTTP Access tokens" configuration.  The required
token permissions for \cxoneflow operations for any HTTP access token are:

\begin{itemize}
    \item Project read
    \item Repository read
    \item Repository write
\end{itemize}


\begin{figure}[ht]
    \includegraphics[width=\textwidth]{graphics/bbdc-token-config.png}
    \caption{BitBucket Data Center Project-Level HTTP Access Token Config}
    \label{fig:bbdc-token-config}
\end{figure}


User HTTP access tokens are tied to a user account but are not subject to password
changes, 2FA authentication, and captcha challenges with interactive logins.  A token generated
from a user as a service account can be granted the required permissions across multiple projects.
In this scenario, a single SCM service configuration would be able to handle events coming from
repositories in multiple projects.

\section{\cxoneflowtext\space SSH Keys}

While performing scan orchestration, \cxoneflow does access the BitBucket Data Center API for
certain operations.  This requires a configuration in the \texttt{api-auth} configuration
element as described in Section \ref{sec:yaml-config}.  The \texttt{clone-auth},
described in Section \ref{sec:yaml-config}, is an optional element where the credentials
used for cloning code can be provided.  If \texttt{clone-auth} is not provided, cloning will
be attempted using the credentials defined by \texttt{api-auth}.

The \texttt{clone-auth} configuration can define an SSH private key for use in cloning.  This
will allow for a separate set of credentials or authentication methods between cloning and
API use.


\section{Protected Branches}

The \cxoneflow workflow, as described in Section \ref{sec:overview}, uses the concept of "Protected Branches"
to know when to invoke workflows.  BitBucket Data Center allows for the configuration of the branching model
at the project and repository level.  Some repositories inherit their branching model from the project
configuration, but the ability for this to be overridden at the repository level is an optional configuration.
The branching model applied to the repository that emitted the webhook event is used to determine which branches
are "Protected Branches" at the time \cxoneflow handles the event.

The project-level branching model configuration is shown in Figure \ref{fig:bbdc-branch-config}.  The
repository-level branching model configuration is similar in that both allow the definition of
"Development" and "Production" branches.  \cxoneflow considers any branch specified as a Development
or Production branch to be a "Protected Branch".

\begin{figure}[ht]
    \includegraphics[width=\textwidth]{graphics/bbdc-branch-config.png}
    \caption{BitBucket Data Center Project-Level Branch Config}
    \label{fig:bbdc-branch-config}
\end{figure}



\section{Webhook Configuration}

Figure \ref{fig:bbdc-project-config} shows the BitBucket Data Center project configuration screen.  The
project key will appear in clone URLs and can be used as part of the regular expression 
placed in the \texttt{repo-match} configuration element.  Please see Section \ref{sec:yaml-config} 
for a description of the \texttt{repo-match} configuration element. The project's "Webhooks" configuration
can be used to configure the \cxoneflow endpoint \texttt{/bbdc} to receive webhook events for each repository in the organization.  

\begin{figure}[ht]
    \includegraphics[width=\textwidth]{graphics/bbdc-project-config.png}
    \caption{BitBucket Data Center Project Configuration}
    \label{fig:bbdc-project-config}
\end{figure}

\begin{figure}[ht]
    \includegraphics[width=\textwidth]{graphics/bbdc-webhook-config.png}
    \caption{BitBucket Data Center Webhook Configuration}
    \label{fig:bbdc-webhook-config}
\end{figure}


Figure \ref{fig:bbdc-webhook-config} shows a typical webhook configuration.  The \textit{Secret} is how \cxoneflow
validates the origin of the event payload.  The configuration element \texttt{shared-secret}, as described
in Section \ref{sec:yaml-config}, should be configured with the webhook secret value.  If \cxoneflow
is running at the specified URL endpoint, the "Test Connection" button will send a diagnostic ping
and receive back a positive response.  If the connection test fails, please ensure that \cxoneflow is running
at the address specified in the URL field and that the BitBucket Data Center server can make a connection
to that URL.

If the webhook is configured at the Project scope, the events sent apply to all repositories contained
within the project.  Figure \ref{fig:bbdc-repo-event-config} shows the configured repository-level webhook 
events that will send a webhook payload to the \cxoneflow endpoint. 
Figure \ref{fig:bbdc-pr-event-config} shows the configured pull-request events that will be sent to 
the \cxoneflow endpoint.  The following events are currently supported:

\begin{itemize}
    \item Repository -> Push
    \item Pull Request -> Opened
    \item Pull Request -> Source branch updated
    \item Pull Request -> Modified
    \item Pull Request -> Approved
    \item Pull Request -> Changes requested
    \item Pull Request -> Declined
    \item Pull Request -> Unapproved
    \item Pull Request -> Merged
    \item Pull Request -> Deleted
\end{itemize}


\begin{figure}[ht]
    \includegraphics[width=\textwidth]{graphics/bbdc-repository-event-config.png}
    \caption{BitBucket Data Center Webhook Repository Event Config}
    \label{fig:bbdc-repo-event-config}
\end{figure}

\begin{figure}[ht]
    \includegraphics[width=\textwidth]{graphics/bbdc-pr-event-config.png}
    \caption{BitBucket Data Center Webhook Pull Request Event Config}
    \label{fig:bbdc-pr-event-config}
\end{figure}


\section{\cxoneflowtext\space HTTP Access Tokens}

Current versions of BitBucket Data Center now reject the use of Basic Authorization using
a username and password. While it may be possible to use Basic Authorization to access the SCM, typically
this is a configuration that should be avoided.  In addition to Basic Authorization breaking upon BitBucket Data Center
server upgrades, an interactive user account may be subject to password changes and Captcha verification that
can break \cxoneflow operations.  

It is generally best to utilize a user HTTP access token for SCM connection
configurations \texttt{api-auth}.  If a user HTTP access token is used, the user can be granted
permissions for multiple projects; this will make the \cxoneflow configuration much easier. 

BitBucket Data Center can create project HTTP access tokens that are similar in use to
a user HTTP access token.  A project HTTP access token is not tied to a specific user, but the
access scope is limited to accessing repositories in the project where it was generated.
If using project HTTP access tokens, one project HTTP access token is required for \cxoneflow
to handle events emitted from repositories in each corresponding project. This also implies that
the \cxoneflow configuration will require one SCM service configuration per project with an
associated \texttt{repo-match} regular expression so the correct project token is used when
orchestrating the scan.

Please refer to Section \ref{sec:yaml-config} for more details about using the HTTP access token
to configure the BitBucket Data Center connection.

Figure \ref{fig:bbdc-token-config} shows the project "HTTP Access tokens" configuration.  The required
token permissions for \cxoneflow operations for any HTTP access token are:

\begin{itemize}
    \item Project read
    \item Repository read
    \item Repository write
\end{itemize}


\begin{figure}[ht]
    \includegraphics[width=\textwidth]{graphics/bbdc-token-config.png}
    \caption{BitBucket Data Center Project-Level HTTP Access Token Config}
    \label{fig:bbdc-token-config}
\end{figure}


User HTTP access tokens are tied to a user account but are not subject to password
changes, 2FA authentication, and captcha challenges with interactive logins.  A token generated
from a user as a service account can be granted the required permissions across multiple projects.
In this scenario, a single SCM service configuration would be able to handle events coming from
repositories in multiple projects.

\section{\cxoneflowtext\space SSH Keys}

While performing scan orchestration, \cxoneflow does access the BitBucket Data Center API for
certain operations.  This requires a configuration in the \texttt{api-auth} configuration
element as described in Section \ref{sec:yaml-config}.  The \texttt{clone-auth},
described in Section \ref{sec:yaml-config}, is an optional element where the credentials
used for cloning code can be provided.  If \texttt{clone-auth} is not provided, cloning will
be attempted using the credentials defined by \texttt{api-auth}.

The \texttt{clone-auth} configuration can define an SSH private key for use in cloning.  This
will allow for a separate set of credentials or authentication methods between cloning and
API use.


\section{Protected Branches}

The \cxoneflow workflow, as described in Section \ref{sec:overview}, uses the concept of "Protected Branches"
to know when to invoke workflows.  BitBucket Data Center allows for the configuration of the branching model
at the project and repository level.  Some repositories inherit their branching model from the project
configuration, but the ability for this to be overridden at the repository level is an optional configuration.
The branching model applied to the repository that emitted the webhook event is used to determine which branches
are "Protected Branches" at the time \cxoneflow handles the event.

The project-level branching model configuration is shown in Figure \ref{fig:bbdc-branch-config}.  The
repository-level branching model configuration is similar in that both allow the definition of
"Development" and "Production" branches.  \cxoneflow considers any branch specified as a Development
or Production branch to be a "Protected Branch".

\begin{figure}[ht]
    \includegraphics[width=\textwidth]{graphics/bbdc-branch-config.png}
    \caption{BitBucket Data Center Project-Level Branch Config}
    \label{fig:bbdc-branch-config}
\end{figure}



\subsection{Azure DevOps Webhook Topology}

Azure DevOps Enterprise (on-premise) uses one or more \textbf{Collection} logical units to separate
repositories into logical groups.  A \textbf{Collection} in Azure DevOps Enterprise 
corresponds to a Azure DevOps Cloud \textbf{Organization} in that the \textbf{Organization} is a logical unit 
that separates repositories into logical groups. Web hook deployments are not available at this scope.

In each \textbf{Collection} or \textbf{Organization}, zero or more \textbf{Project} logical
units establish the next level of repository organization.  Each \textbf{Project} will 
have one or more \textbf{Repository} units that will contain code that should be scanned.  Webhooks
can be deployed at this scope; each \textbf{Repository} in the configured \textbf{Project} logical unit
will emit webhook events when Service Hooks have been configured in the \textbf{Project} to
deliver webhook events to \cxoneflow.

Configuration of Service Hooks at the \textbf{Project} scope is generally the preferred
method of webhook deployment for Azure DevOps.

Webhooks can be deployed at the scope of each \textbf{Repository} if desired.  The number of repositories
in a large enterprise generally makes deployment at the \textbf{Repository} scope useful
only for testing purposes.


\subsection{GitHub Webhook Topology}

GitHub uses one or more \textbf{Organization} logical units to separate zero or more
repositories into logical groupings. Each GitHub user is also a limited variation of an
\textbf{Organization} in that a user account is also a logical unit that can have
associated repositories.  Webhook configurations can be deployed at this scope.
When webhooks are deployed at the \textbf{Organization} scope, events
will be emitted for all repositories within the \textbf{Organization} logical unit.

Configuration of webhooks at the \textbf{Organization} scope is generally the preferred
method of webhook deployment for GitHub.

A \textbf{GitHub App} can be created that functions as a webhook deployment template.
The \textbf{GitHub App} can be deployed at the \textbf{Organization} scope using the
GitHub user interface.  This is the preferred method of defining and executing the webhook deployment.

Webhooks can be deployed at the scope of each \textbf{Repository} if desired.  The number of repositories
in a large enterprise generally makes deployment at the \textbf{Repository} scope useful
only for testing purposes.  


\subsection{Gitlab Webhook Topology}

\chapter{Self-Hosted Gitlab and Gitlab Cloud}


\section{About Gitlab}

The Gitlab integration supports both self-hosted and cloud-hosted Gitlab instances.  Each type
of Gitlab instance uses a subscription model that can limit access to some of the webhook
deployment features that can be utilized by \cxoneflowns. The effort to deploy \cxoneflow at scale
across a large number of repositories can change based upon the level of subscription
and how the Gitlab instance is hosted.

The \cxoneflow endpoint \texttt{/gl} is the handler for all webhook event
payloads originating from any Gitlab instance type.  


\section{Webhook Topology}

\chapter{Self-Hosted Gitlab and Gitlab Cloud}


\section{About Gitlab}

The Gitlab integration supports both self-hosted and cloud-hosted Gitlab instances.  Each type
of Gitlab instance uses a subscription model that can limit access to some of the webhook
deployment features that can be utilized by \cxoneflowns. The effort to deploy \cxoneflow at scale
across a large number of repositories can change based upon the level of subscription
and how the Gitlab instance is hosted.

The \cxoneflow endpoint \texttt{/gl} is the handler for all webhook event
payloads originating from any Gitlab instance type.  


\section{Webhook Topology}

\chapter{Self-Hosted Gitlab and Gitlab Cloud}


\section{About Gitlab}

The Gitlab integration supports both self-hosted and cloud-hosted Gitlab instances.  Each type
of Gitlab instance uses a subscription model that can limit access to some of the webhook
deployment features that can be utilized by \cxoneflowns. The effort to deploy \cxoneflow at scale
across a large number of repositories can change based upon the level of subscription
and how the Gitlab instance is hosted.

The \cxoneflow endpoint \texttt{/gl} is the handler for all webhook event
payloads originating from any Gitlab instance type.  


\section{Webhook Topology}

\input{scms/topology/gl.tex}


\section{\cxoneflowtext\space YAML Configuration for GitLab}

Below is a listing of a minimal Gitlab YAML configuration.  The elements to note in this
configuration:

\begin{itemize}
    \item The \texttt{base-url} is the root URL for the Gitlab instance.  This URL is also used
    when forming display links such as links to code in PR comments.
    \item The \texttt{api-url-suffix} is the path extension appended to \texttt{base-url} for API requests.
    \item The \texttt{repo-match} regular expression matches the service definition with events emitted by any repository
    in the Gitlab instance.
\end{itemize}

\input{operation/gl_minimal_example.tex}

\section{Deploying Webhook Configurations in Gitlab}

The scope where webhooks are deployed for \cxoneflow will determine how to provision a PAT
that has the appropriate permissions to the projects in the deployment scope.  Using service
hooks, for example, requires a PAT from a user that has access to all projects in the Gitlab
instance.

Deploying webhooks at the project or group scope requires a PAT from a user that has access
to the projects in the deployed scope.  This may require multiple \cxoneflow service definitions
with the \texttt{repo-match} regular expression such that events in the scope available to a
PAT are handled properly.

A PAT, regardless of user or scope, requires the following permissions:

\begin{itemize}
  \item api
  \item api\_read
  \item ai features
\end{itemize}

\subsection{Service Hooks}

A Gitlab administrator can configure system-wide service hooks.  The service hooks configuration is located in the
administrative configuration menu, accessible by clicking the "Admin" button as shown in Figure \ref{fig:gl-admin-menu}.

\begin{figure}[ht]
  \centering
  \includegraphics[scale=.5]{graphics/gl-system-hook-admin.png}
  \caption{Gitlab System Hook Administration Menu}
  \label{fig:gl-admin-menu}
\end{figure}


An example service hook definition is shown in Figure \ref{fig:gl-service-hook-def}.  The following are
required configuration elements:

\begin{itemize}
  \item The URL to the \cxoneflow instance ending with the \texttt{gl} route.
  \item The value provided in the \texttt{shared-secret} element under the \texttt{connection} element in the service definition.
  \item The triggers should be \textbf{Push events} and \textbf{Merge request events}.
\end{itemize}

Click the "Add webhook" button to deploy the service hook definition.  Gitlab will immediately start delivering
events to the configured \cxoneflow endpoint.

\begin{figure}[ht]
  \centering
  \includegraphics[width=\textwidth]{graphics/gl-service-hooks.png}
  \caption{Gitlab System Hook Definition}
  \label{fig:gl-service-hook-def}
\end{figure}

\subsection{Group or Project Webhooks}

Webhook configurations at the group or project scope operate the same.  The settings for each
are accessed by navigating to the group or project view and selecting \textbf{Settings->Webhooks}.
If attempting to configure webhooks at the group scope in a free Gitlab self-hosted instance or
a free Gitlab cloud account, you will be prompted to purchase a license.

The webhook configuration can be initiated by clicking the \textbf{Add new webhook} button. Figure \ref{fig:gl-project-1}
shows the endpoint configuration where the URL for the \cxoneflow instance with the \texttt{gl} is provided
along with the shared secret.  The shared secret value should match the value provided in the
\texttt{shared-secret} element under the \texttt{connection} element in the service definition.

\begin{figure}[ht]
  \centering
  \includegraphics[width=\textwidth]{graphics/gl-project-hook-1.png}
  \caption{Gitlab Group/Project Webhook Endpoint Configuration}
  \label{fig:gl-project-1}
\end{figure}

Figure \ref{fig:gl-project-2} shows the webhook configured to send events for 
\textbf{Push events} and \textbf{Merge request events}.  While it is possible to filter
which branches emit push events, doing so may cause \cxoneflow to not work as expected.

Click the "Add webhook" button to deploy the webhook definition.  Gitlab will immediately start delivering
events to the configured \cxoneflow endpoint.


\begin{figure}[ht]
  \centering
  \includegraphics[width=\textwidth]{graphics/gl-project-hook-2.png}
  \caption{Gitlab Group/Project Webhook Event Configuration}
  \label{fig:gl-project-2}
\end{figure}



\section{Protected Branches}

Gitlab allows for a group scope definition of a "default branch" that can be overridden at the
project scope.  The default branch may be considered a protected branch (e.g. it limits who can
commit to the default branch), but Gitlab does not require the default branch to be a protected
branch.  For consistency with \cxoneflow workflow logic with other SCMs, the default branch is
considered a protected branch for all event handling workflows.  This means a push to the 
default branch or a merge request targeting the default branch will initiate a scan regardless
of the configured protection status.

Branch protection rules other than those associated with the default branch can be configured at
the project scope.  The rules allow a named branch to be configured as protected or to supply
a wildcard to protect branches with names that match the wildcard specification.  \cxoneflow will
follow the Gitlab configuration logic for protected branches when handling an event.  Any push
or merge request targeting a protected branch will initiate a scan.



\section{\cxoneflowtext\space YAML Configuration for GitLab}

Below is a listing of a minimal Gitlab YAML configuration.  The elements to note in this
configuration:

\begin{itemize}
    \item The \texttt{base-url} is the root URL for the Gitlab instance.  This URL is also used
    when forming display links such as links to code in PR comments.
    \item The \texttt{api-url-suffix} is the path extension appended to \texttt{base-url} for API requests.
    \item The \texttt{repo-match} regular expression matches the service definition with events emitted by any repository
    in the Gitlab instance.
\end{itemize}

\begin{code}{Minimal YAML Configuration Example}{Gitlab Webhooks}{}
  secret-root-path: /run/secrets
  server-base-url: https://cxoneflow.mydomain.com:8443/
  
  gh:
      - service-name: Gitlab
        repo-match: ^http(s)?:(\/){2}gitlab\.corp\.com.*
        feedback:
          pull-request:
            enabled: True
        connection:
          base-url: https://gitlab.corp.com
          api-url-suffix: api/v4
          shared-secret: scm-shared-secret
          api-auth:
            token: ghe-token
        cxone:
          tenant: mytenant
          oauth:
            client-id: my-oauth-id
            client-secret: my-oauth-secret
          iam-endpoint: US
          api-endpoint: US
\end{code}
    

\section{Deploying Webhook Configurations in Gitlab}

The scope where webhooks are deployed for \cxoneflow will determine how to provision a PAT
that has the appropriate permissions to the projects in the deployment scope.  Using service
hooks, for example, requires a PAT from a user that has access to all projects in the Gitlab
instance.

Deploying webhooks at the project or group scope requires a PAT from a user that has access
to the projects in the deployed scope.  This may require multiple \cxoneflow service definitions
with the \texttt{repo-match} regular expression such that events in the scope available to a
PAT are handled properly.

A PAT, regardless of user or scope, requires the following permissions:

\begin{itemize}
  \item api
  \item api\_read
  \item ai features
\end{itemize}

\subsection{Service Hooks}

A Gitlab administrator can configure system-wide service hooks.  The service hooks configuration is located in the
administrative configuration menu, accessible by clicking the "Admin" button as shown in Figure \ref{fig:gl-admin-menu}.

\begin{figure}[ht]
  \centering
  \includegraphics[scale=.5]{graphics/gl-system-hook-admin.png}
  \caption{Gitlab System Hook Administration Menu}
  \label{fig:gl-admin-menu}
\end{figure}


An example service hook definition is shown in Figure \ref{fig:gl-service-hook-def}.  The following are
required configuration elements:

\begin{itemize}
  \item The URL to the \cxoneflow instance ending with the \texttt{gl} route.
  \item The value provided in the \texttt{shared-secret} element under the \texttt{connection} element in the service definition.
  \item The triggers should be \textbf{Push events} and \textbf{Merge request events}.
\end{itemize}

Click the "Add webhook" button to deploy the service hook definition.  Gitlab will immediately start delivering
events to the configured \cxoneflow endpoint.

\begin{figure}[ht]
  \centering
  \includegraphics[width=\textwidth]{graphics/gl-service-hooks.png}
  \caption{Gitlab System Hook Definition}
  \label{fig:gl-service-hook-def}
\end{figure}

\subsection{Group or Project Webhooks}

Webhook configurations at the group or project scope operate the same.  The settings for each
are accessed by navigating to the group or project view and selecting \textbf{Settings->Webhooks}.
If attempting to configure webhooks at the group scope in a free Gitlab self-hosted instance or
a free Gitlab cloud account, you will be prompted to purchase a license.

The webhook configuration can be initiated by clicking the \textbf{Add new webhook} button. Figure \ref{fig:gl-project-1}
shows the endpoint configuration where the URL for the \cxoneflow instance with the \texttt{gl} is provided
along with the shared secret.  The shared secret value should match the value provided in the
\texttt{shared-secret} element under the \texttt{connection} element in the service definition.

\begin{figure}[ht]
  \centering
  \includegraphics[width=\textwidth]{graphics/gl-project-hook-1.png}
  \caption{Gitlab Group/Project Webhook Endpoint Configuration}
  \label{fig:gl-project-1}
\end{figure}

Figure \ref{fig:gl-project-2} shows the webhook configured to send events for 
\textbf{Push events} and \textbf{Merge request events}.  While it is possible to filter
which branches emit push events, doing so may cause \cxoneflow to not work as expected.

Click the "Add webhook" button to deploy the webhook definition.  Gitlab will immediately start delivering
events to the configured \cxoneflow endpoint.


\begin{figure}[ht]
  \centering
  \includegraphics[width=\textwidth]{graphics/gl-project-hook-2.png}
  \caption{Gitlab Group/Project Webhook Event Configuration}
  \label{fig:gl-project-2}
\end{figure}



\section{Protected Branches}

Gitlab allows for a group scope definition of a "default branch" that can be overridden at the
project scope.  The default branch may be considered a protected branch (e.g. it limits who can
commit to the default branch), but Gitlab does not require the default branch to be a protected
branch.  For consistency with \cxoneflow workflow logic with other SCMs, the default branch is
considered a protected branch for all event handling workflows.  This means a push to the 
default branch or a merge request targeting the default branch will initiate a scan regardless
of the configured protection status.

Branch protection rules other than those associated with the default branch can be configured at
the project scope.  The rules allow a named branch to be configured as protected or to supply
a wildcard to protect branches with names that match the wildcard specification.  \cxoneflow will
follow the Gitlab configuration logic for protected branches when handling an event.  Any push
or merge request targeting a protected branch will initiate a scan.



\section{\cxoneflowtext\space YAML Configuration for GitLab}

Below is a listing of a minimal Gitlab YAML configuration.  The elements to note in this
configuration:

\begin{itemize}
    \item The \texttt{base-url} is the root URL for the Gitlab instance.  This URL is also used
    when forming display links such as links to code in PR comments.
    \item The \texttt{api-url-suffix} is the path extension appended to \texttt{base-url} for API requests.
    \item The \texttt{repo-match} regular expression matches the service definition with events emitted by any repository
    in the Gitlab instance.
\end{itemize}

\begin{code}{Minimal YAML Configuration Example}{Gitlab Webhooks}{}
  secret-root-path: /run/secrets
  server-base-url: https://cxoneflow.mydomain.com:8443/
  
  gh:
      - service-name: Gitlab
        repo-match: ^http(s)?:(\/){2}gitlab\.corp\.com.*
        feedback:
          pull-request:
            enabled: True
        connection:
          base-url: https://gitlab.corp.com
          api-url-suffix: api/v4
          shared-secret: scm-shared-secret
          api-auth:
            token: ghe-token
        cxone:
          tenant: mytenant
          oauth:
            client-id: my-oauth-id
            client-secret: my-oauth-secret
          iam-endpoint: US
          api-endpoint: US
\end{code}
    

\section{Deploying Webhook Configurations in Gitlab}

The scope where webhooks are deployed for \cxoneflow will determine how to provision a PAT
that has the appropriate permissions to the projects in the deployment scope.  Using service
hooks, for example, requires a PAT from a user that has access to all projects in the Gitlab
instance.

Deploying webhooks at the project or group scope requires a PAT from a user that has access
to the projects in the deployed scope.  This may require multiple \cxoneflow service definitions
with the \texttt{repo-match} regular expression such that events in the scope available to a
PAT are handled properly.

A PAT, regardless of user or scope, requires the following permissions:

\begin{itemize}
  \item api
  \item api\_read
  \item ai features
\end{itemize}

\subsection{Service Hooks}

A Gitlab administrator can configure system-wide service hooks.  The service hooks configuration is located in the
administrative configuration menu, accessible by clicking the "Admin" button as shown in Figure \ref{fig:gl-admin-menu}.

\begin{figure}[ht]
  \centering
  \includegraphics[scale=.5]{graphics/gl-system-hook-admin.png}
  \caption{Gitlab System Hook Administration Menu}
  \label{fig:gl-admin-menu}
\end{figure}


An example service hook definition is shown in Figure \ref{fig:gl-service-hook-def}.  The following are
required configuration elements:

\begin{itemize}
  \item The URL to the \cxoneflow instance ending with the \texttt{gl} route.
  \item The value provided in the \texttt{shared-secret} element under the \texttt{connection} element in the service definition.
  \item The triggers should be \textbf{Push events} and \textbf{Merge request events}.
\end{itemize}

Click the "Add webhook" button to deploy the service hook definition.  Gitlab will immediately start delivering
events to the configured \cxoneflow endpoint.

\begin{figure}[ht]
  \centering
  \includegraphics[width=\textwidth]{graphics/gl-service-hooks.png}
  \caption{Gitlab System Hook Definition}
  \label{fig:gl-service-hook-def}
\end{figure}

\subsection{Group or Project Webhooks}

Webhook configurations at the group or project scope operate the same.  The settings for each
are accessed by navigating to the group or project view and selecting \textbf{Settings->Webhooks}.
If attempting to configure webhooks at the group scope in a free Gitlab self-hosted instance or
a free Gitlab cloud account, you will be prompted to purchase a license.

The webhook configuration can be initiated by clicking the \textbf{Add new webhook} button. Figure \ref{fig:gl-project-1}
shows the endpoint configuration where the URL for the \cxoneflow instance with the \texttt{gl} is provided
along with the shared secret.  The shared secret value should match the value provided in the
\texttt{shared-secret} element under the \texttt{connection} element in the service definition.

\begin{figure}[ht]
  \centering
  \includegraphics[width=\textwidth]{graphics/gl-project-hook-1.png}
  \caption{Gitlab Group/Project Webhook Endpoint Configuration}
  \label{fig:gl-project-1}
\end{figure}

Figure \ref{fig:gl-project-2} shows the webhook configured to send events for 
\textbf{Push events} and \textbf{Merge request events}.  While it is possible to filter
which branches emit push events, doing so may cause \cxoneflow to not work as expected.

Click the "Add webhook" button to deploy the webhook definition.  Gitlab will immediately start delivering
events to the configured \cxoneflow endpoint.


\begin{figure}[ht]
  \centering
  \includegraphics[width=\textwidth]{graphics/gl-project-hook-2.png}
  \caption{Gitlab Group/Project Webhook Event Configuration}
  \label{fig:gl-project-2}
\end{figure}



\section{Protected Branches}

Gitlab allows for a group scope definition of a "default branch" that can be overridden at the
project scope.  The default branch may be considered a protected branch (e.g. it limits who can
commit to the default branch), but Gitlab does not require the default branch to be a protected
branch.  For consistency with \cxoneflow workflow logic with other SCMs, the default branch is
considered a protected branch for all event handling workflows.  This means a push to the 
default branch or a merge request targeting the default branch will initiate a scan regardless
of the configured protection status.

Branch protection rules other than those associated with the default branch can be configured at
the project scope.  The rules allow a named branch to be configured as protected or to supply
a wildcard to protect branches with names that match the wildcard specification.  \cxoneflow will
follow the Gitlab configuration logic for protected branches when handling an event.  Any push
or merge request targeting a protected branch will initiate a scan.



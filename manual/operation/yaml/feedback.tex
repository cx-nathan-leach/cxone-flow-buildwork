\subsubsection{YAML Element: <scm moniker>.feedback.pull-request}\label{sec:yaml-feedback-pull-request}
The configuration parameters for pull request feedback workflows.

\subsubsection{YAML Element: <scm moniker>.feedback.pull-request.enabled}\label{sec:yaml-pull-request-enabled}
Defaults to \texttt{False}.  If set to \texttt{True}, the feedback workflow for Pull Requests is executed upon completion of a scan generated by
a Pull Request. See Section \ref{sec:pull-request-workflow} for details about the Pull Request feedback workflow.


\subsubsection{YAML Element: <scm moniker>.feedback.push}\label{sec:yaml-feedback-push}
The configuration parameters for Sarif delivery as push feedback.

\subsubsection{YAML Element: <scm moniker>.feedback.push.enabled}\label{sec:yaml-push-enabled}
Defaults to \texttt{False}.  If set to \texttt{True}, delivery of Sarif logs is executed upon completion of a scan
of a protected branch. See Section \ref{sec:push-workflow} for details about the Pull Request feedback workflow.

\subsubsection{YAML Element: <scm moniker>.feedback.push.sarif-opts}\label{sec:yaml-push-sarif-opts}
This is a dictionary that follows the structure of the \extlink{https://github.com/checkmarx-ts/cxone-sarif/blob/master/cxone_sarif/opts/__init__.py}{ReportOpts}
object in the \cxonesarif Python module.  This defines the options used when generating the Sarif log.

\subsubsection{YAML Element: <scm moniker>.feedback.push.sarif-opts.SastOpts}\label{sec:yaml-push-sarif-opts-sastopts}
A dictionary that contains elements that control the options used for SAST options in the Sarif dictionary.

\subsubsection{YAML Element: <scm moniker>.feedback.push.sarif-opts.SastOpts.OmitApiResults}\label{sec:yaml-push-sarif-opts-sastopts-omitapiresults}
Defaults to \texttt{False}.  If set to \texttt{True}, API security scan results will not be included with the reported SAST results.

\subsubsection{YAML Element: <scm moniker>.feedback.push.sarif-opts.SastOpts.SkipSast}\label{sec:yaml-push-sarif-opts-sastopts-skipsast}
Defaults to \texttt{False}.  If set to \texttt{True}, both SAST and API security scan results will not be included in the generated Sarif log.

\subsubsection{YAML Element: <scm moniker>.feedback.push.sarif-opts.SkipContainers}\label{sec:yaml-push-sarif-opts-skipcontainers}
Defaults to \texttt{False}.  If set to \texttt{True}, Container scan results will not be included in the generated Sarif log.
\subsubsection{YAML Element: <scm moniker>.feedback.push.sarif-opts.SkipKics}\label{sec:yaml-push-sarif-opts-skipkics}
Defaults to \texttt{False}.  If set to \texttt{True}, KICS (IaC) scan results will not be included in the generated Sarif log.
\subsubsection{YAML Element: <scm moniker>.feedback.push.sarif-opts.SkipSca}\label{sec:yaml-push-sarif-opts-skipsca}
Defaults to \texttt{False}.  If set to \texttt{True}, SCA scan results will not be included in the generated Sarif log.

\subsubsection{YAML Element: <scm moniker>.feedback.push.via-amqp}\label{sec:yaml-push-via-amqp}
A dictionary of configuration options to define the parameters for publishing the generated Sarif log to an AMQP exchange.

\subsubsection{YAML Element: <scm moniker>.feedback.push.via-amqp.exchange}\label{sec:yaml-push-via-amqp-exchange}
The name of the exchange where the Sarif log message will be submitted.

\subsubsection{YAML Element: <scm moniker>.feedback.push.via-amqp.shared-secret}\label{sec:yaml-push-via-amqp-shared-secret}
A shared secret used for validating an HMAC signature of the Sarif log delivered as a message via AMQP.

\subsubsection{YAML Element: <scm moniker>.feedback.push.via-amqp.topic-prefix}\label{sec:yaml-push-via-amqp-topic-prefix}
An optional value used as a prefix to the service name when forming the topic used when sending the Sarif log to the exchange.

\subsubsection{YAML Element: <scm moniker>.feedback.push.via-amqp.topic-suffix}\label{sec:yaml-push-via-amqp-topic-suffix}
An optional value used as a suffix to the service name when forming the topic used when sending the Sarif log to the exchange.

\subsubsection{YAML Element: <scm moniker>.feedback.push.via-http-post}\label{sec:yaml-push-via-http-post}
A list of one or more groups of configuration elements that define HTTP endpoints where the generated Sarif log will be POSTed.

\subsubsection{YAML Element: <scm moniker>.feedback.push.via-http-post.delivery-retries}\label{sec:yaml-push-via-http-post-delivery-retries}
An integer that defines the number of retries that will be attempted upon failure to deliver a Sarif log to the defined HTTP endpoint.  If not
explicitly defines, the default number of retries is 2.

\subsubsection{YAML Element: <scm moniker>.feedback.push.via-http-post.delivery-retry-delay-seconds}\label{sec:yaml-push-via-http-post-delivery-retry-delay-seconds}
An integer that defines the amount of a time delay, in seconds, between attempts to retry delivery of the Sarif log.  If not explicitly defines, the
delay is 60 seconds.

\subsubsection{YAML Element: <scm moniker>.feedback.push.via-http-post.endpoint-url}\label{sec:yaml-push-via-http-post-endpoint-url}
The URL to the endpoint where the Sarif log will be POSTed.


\subsubsection{YAML Element: <scm moniker>.feedback.push.via-http-post.shared-secret}\label{sec:yaml-push-via-http-post-shared-secret}
A shared secret used for validating an HMAC signature of the Sarif log delivered as the body of a POSTed HTTP request.


\subsubsection{YAML Element: <scm moniker>.feedback.scan-monitor}\label{sec:yaml-feedback-scan-monitor}
The parameters used when monitoring scan progress during workflow orchestration. 

Scan progress is monitored by requesting a scan state from the \cxone API at
a time interval.  The initial time interval is set to the value configured for
\texttt{poll-interval-seconds}.  If the scan is not found to have finished executing
at any given poll execution, the previous poll interval time is multiplied by
the scalar given in the \texttt{poll-backoff-multiplier} value up to a maximum
poll interval time configured by \texttt{poll-max-interval-seconds}.

If a scan does not finish executing by the time set in \texttt{scan-timeout-hours}, the
workflow is aborted.  The value of 0 configured for \texttt{scan-timeout-hours} indicates
the workflow will wait forever for the scan to finish executing.

\subsubsection{YAML Element: <scm moniker>.feedback.exclusions}\label{sec:yaml-feedback-exclusions}
Settings for excluding results from results from feedback output.  Each of the elements is a list that 
can be configured with multiple exclusion elements.

\chapter{Configuration}


\section{Runtime Configuration}\label{sec:runtime-config}

\subsection{SSL}

\subsubsection{Trusting Self-Signed Certificates}\label{sec:self-signed-certs}

While the \cxone system uses TLS certificates signed by a public CA, it is possible that
corporate proxies use certificates signed by a private CA. Private CA certificates must be imported
into the \cxoneflow container.

Each private CA certificate for import must meet the following criteria:

\begin{itemize}
    \item It must be in a file ending with the extension .crt.
    \item The contents of the file must be one certificate stored in the PEM format.
    \item All files containing private CA certificates must be mapped to the container path \texttt{/usr/local/share/ca-certificates}.
\end{itemize}


As an example, if using Docker, it is possible to map a single local file to a file in the container with this mapping 
option added to the container execution command line:

\begin{code}{Custom CA Mapping Option}{[Docker]}{}
-v $(pwd)/custom-ca.pem:/usr/local/share/ca-certificates/custom-ca.crt
\end{code}

\subsubsection{The \texttt{ssl-verify} Option}\label{sec:ssl-verify-general}

In the configuration YAML documentation, all of the \texttt{connection}
elements contain an optional \texttt{ssl-verify} setting.  This option
is generally useful to turn off SSL verification by setting it to \texttt{False}.
This can also be used to control which CA bundle is used for verification.

Omitting the \texttt{ssl-verify} setting should be sufficient for
most deployment cases.  If omitted, the container execution will use the default CA bundle
where any custom CAs are added as described in Section \ref{sec:self-signed-certs}.
The \texttt{ssl-verify} option can be set to an explicit path on the container
if there is a need to use a CA bundle other than the one provided by the OS.


\subsubsection{Configuring SSL for the \cxoneflowtext Endpoint}

Configuring the \cxoneflow endpoint for SSL communication requires an SSL certificate public/private key pair
and map the files to a location on the container.  The following environment variables must then be set in the
runtime environment:

\begin{table}[ht]
    \caption{SSL Environment Variables}
    \begin{tabularx}{\textwidth}{ll}
        \toprule
        \textbf{Variable} & \textbf{Description}\\
        \midrule
        \texttt{SSL\_CERT\_PATH} & \makecell[l]{The path to the server's SSL certificate in PEM format.}\\
        \midrule
        \texttt{SSL\_CERT\_KEY\_PATH} & \makecell[l]{The path to the certificate's unencrypted private key in PEM format.}\\
        \bottomrule
    \end{tabularx}
\end{table}

If your SSL certificate is self-signed, the certificate must also be imported as the CA as described
in Section \ref{sec:self-signed-certs}.  If the certificate is signed with a private CA, the private
CA must also be imported.  Failure to import a non-public signing CA for these types of certificates
will cause \cxoneflow startup failures.


\subsection{Runtime Control Environment Variables}

Environment variables can be set when the \cxoneflow container is executing to control some aspects of \cxoneflowns's operation.
Table \ref{tab:runtime-environment-vars} shows the operational environment variables and their meaning.

\begin{table}[ht]
    \caption{Runtime Control Environment Variables}\label{tab:runtime-environment-vars}
    \begin{tabularx}{\textwidth}{lccl}
        \toprule
        \textbf{Variable} & \textbf{Required} & \textbf{Default} & \textbf{Description}\\
        \midrule
        \texttt{CXONEFLOW\_WORKERS} & No & \texttt{max(\# of CPUs / 2, 1)} & \makecell[l]{The number of worker processes\\used for parallel execution. The\\maximum value will be\\set at \texttt{(\# of CPUs - 1)}}\\
        \midrule
        \texttt{LOG\_LEVEL} & No & \texttt{INFO} & \makecell[l]{The logging verbosity level.  Set to\\\texttt{DEBUG} for increased logging\\verbosity.}\\
        \midrule
        \texttt{CONFIG\_YAML\_PATH} & No & \texttt{/opt/cxone/config.yaml} & \makecell[l]{The path to the configuration\\YAML file.}\\
        \midrule
        \texttt{CXONEFLOW\_HOSTNAME} & No & \texttt{localhost} & \makecell[l]{The virtual hostname of the\\\cxoneflow endpoint.}\\
        \bottomrule
    \end{tabularx}
\end{table}


\newpage

\section{Operational Configuration}\label{sec:op-config}

The operational configuration uses a YAML file mapped at \texttt{/opt/cxone/config.yaml}
by default.  It is possible to map the \texttt{config.yaml} file to another location in the
container and adjust the path via the \texttt{CONFIG\_YAML\_PATH} environment variable.

\subsection{YAML Configuration Examples}

Starter YAML example files are now included as a downloadable archive found with the \cxoneflow GitHub
release artifacts.  The configuration files can be customized to fit your runtime requirements or used
as a reference to create new service definitions.

The configuration file is organized into lists of service definitions that apply to each supported
SCM type.  A supported SCM may have one or more service definitions to support variations of
orchestrating scan workflows.  Many service definitions will have repeated elements (such as
the \cxone connection definition); to avoid duplicating elements, 
\extlink{https://docs.docker.com/compose/compose-file/10-fragments/}{YAML Anchors}
can be used to avoid repetitive elements across service definitions.  


\subsection{YAML Configuration Elements}\label{sec:yaml-config}

The organization of the YAML configuration is depicted in the tree below.  The description of each element
can be referenced by clicking the element.  Required elements are indicated in the tree; in general, if an
element that is not marked "required" is omitted, the feature that performs that operation is not invoked
for the configured service definition.

The \texttt{<root>} element indicates that elements directly under the root start at the farthest
left index of the line (this means a line position with an index of 0).  YAML elements that appear
under a parent element are intended to start at first tab stop past the parent element's tab stop.
Anchor elements may be defined at the root but must not clash with the names of any of the root elements.

Parts of the YAML tree have been split into individual trees to allow related elements to appear together.

\paragraph{YAML Root Elements}\label{sec:yaml-root}

\noindent\\

\dirtree{%
    .1 <root>.
    .2 \intlink{sec:yaml-script-path}{script-path} \DTcomment{[Optional]}.
    .2 \intlink{sec:yaml-secret-root-path}{secret-root-path} \DTcomment{[Required]}.
    .2 \intlink{sec:yaml-server-base-url}{server-base-url} \DTcomment{[Required]}.
    .2 \intlink{sec:yaml-scm-monikers}{<scm moniker>} \DTcomment{[At least 1 required: \textbf{bbdc}, \textbf{adoe}, \textbf{gh}, \textbf{gl}]}.
    .3 \intlink{sec:moniker-elements}{...see "YAML SCM Moniker Elements"}.
}

\pagebreak
\paragraph{YAML SCM Moniker Elements}\label{sec:moniker-elements}
\noindent\\The \texttt{<scm moniker>} element is a YAML list of dictionaries.  For a YAML list,
each entry is indented to the next tab after the parent and prefixed with a "\texttt{-}" (dash).
The elements in each list entry under \texttt{<scm moniker>} define key/value dictionary entries 
as the list entry.  Each list entry is referred to as a "service definition"
elsewhere in this document.\\\\

\dirtree{%
    .1 <root>.
    .2 \intlink{sec:yaml-scm-monikers}{<scm moniker>} \DTcomment{[Limited to: \textbf{bbdc}, \textbf{adoe}, \textbf{gh}, \textbf{gl}]}.
    .3 \intlink{sec:yaml-moniker-connection}{connection} \DTcomment{[Required]}.
    .4 \intlink{sec:connection-elements}{...see "YAML \texttt{connection} Elements"}.
    .3 \intlink{sec:yaml-moniker-cxone}{cxone} \DTcomment{[Required]}.
    .4 \intlink{sec:cxone-elements}{...see "YAML \texttt{cxone} Elements"}.
    .3 \intlink{sec:yaml-moniker-feedback}{feedback} \DTcomment{[Optional]}.
    .4 \intlink{sec:feedback-elements}{...see "YAML \texttt{feedback} Elements"}.
    .3 \intlink{sec:yaml-moniker-kickoff}{kickoff} \DTcomment{[Optional]}.
    .4 \intlink{sec:kickoff-elements}{...see "YAML \texttt{kickoff} Elements"}.
    .3 \intlink{sec:yaml-moniker-project-groups}{project-groups} \DTcomment{[Optional]}.
    .4 \intlink{sec:project-groups-elements}{...see "YAML \texttt{project-groups} Elements"}.
    .3 \intlink{sec:yaml-moniker-project-naming}{project-naming} \DTcomment{[Optional]}.
    .4 \intlink{sec:project-naming-elements}{...see "YAML \texttt{project-naming} Elements"}.
    .3 \intlink{sec:yaml-moniker-repo-match}{repo-match} \DTcomment{[Required]}.
    .3 \intlink{sec:yaml-moniker-resolver}{resolver} \DTcomment{[Optional]}.
    .4 \intlink{sec:resolver-elements}{...see "YAML \texttt{resolver} Elements"}.
    .3 \intlink{sec:yaml-moniker-scan-config}{scan-config} \DTcomment{[Optional]}.
    .4 \intlink{sec:scan-config-elements}{...see "YAML \texttt{scan-config} Elements"}.
    .3 \intlink{sec:yaml-moniker-service-name}{service-name} \DTcomment{[Required]}.
}

\pagebreak
\paragraph{YAML \texttt{connection} Elements}\label{sec:connection-elements}
\noindent\\

\dirtree{%
    .1 <root>.
    .2 \intlink{sec:yaml-scm-monikers}{<scm moniker>} \DTcomment{[Required: \textbf{bbdc}, \textbf{adoe}, \textbf{gh}, \textbf{gl}]}.
    .3 \intlink{sec:yaml-moniker-connection}{connection} \DTcomment{[Required]}.
    .4 \intlink{sec:yaml-connection-api-auth}{api-auth} \DTcomment{[Required]}.
    .5 \intlink{sec:yaml-api-auth-app-private-key}{app-private-key} \DTcomment{[See element documentation]}.
    .5 \intlink{sec:yaml-api-auth-password}{password} \DTcomment{[See element documentation]}.
    .5 \intlink{sec:yaml-api-auth-token}{token} \DTcomment{[See element documentation]}.
    .5 \intlink{sec:yaml-api-auth-username}{username} \DTcomment{[See element documentation]}.
    .4 \intlink{sec:yaml-connection-api-url-suffix}{api-url-suffix} \DTcomment{[Required for some SCMs]}.
    .4 \intlink{sec:yaml-connection-base-url}{base-url} \DTcomment{[Required]}.
    .4 \intlink{sec:yaml-connection-base-display-url}{base-display-url} \DTcomment{[Required for some SCMs]}.
    .4 \intlink{sec:yaml-connection-clone-auth}{clone-auth} \DTcomment{[Optional] Default: \texttt{api-auth}}.
    .5 \intlink{sec:yaml-api-auth-password}{password} \DTcomment{[See element documentation]}.
    .5 \intlink{sec:yaml-clone-auth-ssh}{ssh} \DTcomment{[See element documentation]}.
    .5 \intlink{sec:yaml-clone-auth-ssh-port}{ssh-port} \DTcomment{[See element documentation]}.
    .5 \intlink{sec:yaml-api-auth-token}{token} \DTcomment{[See element documentation]}.
    .5 \intlink{sec:yaml-api-auth-username}{username} \DTcomment{[See element documentation]}.
    .4 \intlink{sec:yaml-generic-proxies}{proxies} \DTcomment{[Optional]}.
    .4 \intlink{sec:yaml-generic-retries}{retries} \DTcomment{[Optional] Default: 3}.
    .4 \intlink{sec:yaml-connection-shared-secret}{shared-secret} \DTcomment{[Required]}.
    .4 \intlink{sec:yaml-generic-ssl-verify}{ssl-verify} \DTcomment{[Optional] Default: True}.
    .4 \intlink{sec:yaml-generic-timeout-seconds}{timeout-seconds}\DTcomment{[Optional] Default: 60s}.
}

\pagebreak
\paragraph{YAML \texttt{cxone} Elements}\label{sec:cxone-elements}
\noindent\\

\dirtree{%
    .1 <root>.
    .2 \intlink{sec:yaml-scm-monikers}{<scm moniker>} \DTcomment{[Required: \textbf{bbdc}, \textbf{adoe}, \textbf{gh}, \textbf{gl}]}.
    .3 \intlink{sec:yaml-moniker-cxone}{cxone} \DTcomment{[Required]}.
    .4 \intlink{sec:yaml-cxone-api-endpoint}{api-endpoint} \DTcomment{[Required]}.
    .4 \intlink{sec:yaml-cxone-api-key}{api-key} \DTcomment{[Required without oauth]}.
    .4 \intlink{sec:yaml-cxone-iam-endpoint}{iam-endpoint} \DTcomment{[Required]}.
    .4 \intlink{sec:yaml-cxone-oauth}{oauth} \DTcomment{[Required without api-key]}.
    .4 \intlink{sec:yaml-generic-proxies}{proxies} \DTcomment{[Optional]}.
    .4 \intlink{sec:yaml-generic-retries}{retries} \DTcomment{[Optional] Default: 3}.
    .4 \intlink{sec:yaml-generic-ssl-verify}{ssl-verify} \DTcomment{[Optional] Default: True}.
    .4 \intlink{sec:yaml-cxone-tenant}{tenant} \DTcomment{[Required]}.
    .4 \intlink{sec:yaml-generic-timeout-seconds}{timeout-seconds}\DTcomment{[Optional] Default: 60s}.
}


\pagebreak
\paragraph{YAML \texttt{feedback} Elements}\label{sec:feedback-elements}
\noindent\\


\dirtree{%
    .1 <root>.
    .2 \intlink{sec:yaml-scm-monikers}{<scm moniker>} \DTcomment{[Required: \textbf{bbdc}, \textbf{adoe}, \textbf{gh}, \textbf{gl}]}.
    .3 \intlink{sec:yaml-moniker-feedback}{feedback} \DTcomment{[Optional]}.
    .4 \intlink{sec:yaml-generic-amqp}{amqp} \DTcomment{[Optional] Default: container instance}.
    .5 \intlink{sec:yaml-generic-amqp-amqp-password}{amqp-password} \DTcomment{[Optional]}.
    .5 \intlink{sec:yaml-generic-amqp-amqp-url}{amqp-url} \DTcomment{[Required]}.
    .5 \intlink{sec:yaml-generic-amqp-amqp-user}{amqp-user} \DTcomment{[Optional]}.
    .5 \intlink{sec:yaml-generic-ssl-verify}{ssl-verify} \DTcomment{[Optional] Default: True}.
    .4 \intlink{sec:yaml-feedback-pull-request}{pull-request} \DTcomment{[Optional]}.
    .5 \intlink{sec:yaml-pull-request-enabled}{enabled} \DTcomment{[Optional] Default: False}.
    .4 \intlink{sec:yaml-feedback-scan-monitor}{scan-monitor} \DTcomment{[Optional]}.
    .5 \intlink{sec:yaml-scan-monitor-poll-backoff-multiplier}{poll-backoff-multiplier} \DTcomment{[Optional] Default: 2}.
    .5 \intlink{sec:yaml-scan-monitor-poll-interval-seconds}{poll-interval-seconds} \DTcomment{[Optional] Default: 90s}.
    .5 \intlink{sec:yaml-scan-monitor-poll-max-interval-seconds}{poll-max-interval-seconds} \DTcomment{[Optional] Default: 600s}.
    .5 \intlink{sec:yaml-scan-monitor-scan-timeout-hours}{scan-timeout-hours} \DTcomment{[Optional] Default: 48h}.
    .4 \intlink{sec:yaml-feedback-exclusions}{exclusions} \DTcomment{[Optional]}.
    .5 \intlink{sec:yaml-exclusions-severity}{severity} \DTcomment{[Optional]}.
    .5 \intlink{sec:yaml-exclusions-state}{state} \DTcomment{[Optional]}.
}

\pagebreak
\paragraph{YAML \texttt{kickoff} Elements}\label{sec:kickoff-elements}
\noindent\\

\dirtree{%
    .1 <root>.
    .2 \intlink{sec:yaml-scm-monikers}{<scm moniker>} \DTcomment{[Required: \textbf{bbdc}, \textbf{adoe}, \textbf{gh}, \textbf{gl}]}.
    .3 \intlink{sec:yaml-moniker-kickoff}{kickoff} \DTcomment{[Optional]}.
    .4 \intlink{sec:yaml-kickoff-max-concurrent-scans}{max-concurrent-scans} \DTcomment{[Optional]}.
    .4 \intlink{sec:yaml-kickoff-ssh-public-key}{ssh-public-key} \DTcomment{[Required]}.
}

\pagebreak
\paragraph{YAML \texttt{project-groups} Elements}\label{sec:project-groups-elements}
\noindent\\

The \texttt{group-assignments} element is a list containing one or more group matching specifications.
The \texttt{group-assignments.groups} element is a list containing one or more groups to which a project
is assigned.\\


\dirtree{%
    .1 <root>.
    .2 \intlink{sec:yaml-scm-monikers}{<scm moniker>} \DTcomment{[Required: \textbf{bbdc}, \textbf{adoe}, \textbf{gh}, \textbf{gl}]}.
    .3 \intlink{sec:yaml-moniker-project-groups}{project-groups} \DTcomment{[Optional]}.
    .4 \intlink{sec:yaml-project-groups-group-assignments}{group-assignments} \DTcomment{[Required]}.
    .5 \intlink{sec:yaml-project-groups-group-assignments-groups}{groups}.
    .5 \intlink{sec:yaml-project-groups-group-assignments-repo-match}{repo-match}.
    .4 \intlink{sec:yaml-project-groups-update-groups}{update-groups} \DTcomment{[Optional] Default: False}.
}


\pagebreak
\paragraph{YAML \texttt{project-naming} Elements}\label{sec:project-naming-elements}
\noindent\\

\dirtree{%
    .1 <root>.
    .2 \intlink{sec:yaml-scm-monikers}{<scm moniker>} \DTcomment{[Required: \textbf{bbdc}, \textbf{adoe}, \textbf{gh}, \textbf{gl}]}.
    .3 \intlink{sec:yaml-moniker-project-naming}{project-naming} \DTcomment{[Optional]}.
    .4 \intlink{sec:yaml-project-naming-module}{module} \DTcomment{[Optional]}.
    .4 \intlink{sec:yaml-project-naming-update-name}{update-name} \DTcomment{[Optional] Default: False}.
}


\pagebreak
\paragraph{YAML \texttt{resolver} Elements}\label{sec:resolver-elements}
\noindent\\

\dirtree{%
    .1 <root>.
    .2 \intlink{sec:yaml-scm-monikers}{<scm moniker>} \DTcomment{[Required: \textbf{bbdc}, \textbf{adoe}, \textbf{gh}, \textbf{gl}]}.
    .3 \intlink{sec:yaml-moniker-resolver}{resolver} \DTcomment{[Optional]}.
    .4 \intlink{sec:yaml-generic-amqp}{amqp} \DTcomment{[Optional] Default: container instance}.
    .5 \intlink{sec:yaml-generic-amqp-amqp-password}{amqp-password} \DTcomment{[Optional]}.
    .5 \intlink{sec:yaml-generic-amqp-amqp-url}{amqp-url} \DTcomment{[Required]}.
    .5 \intlink{sec:yaml-generic-amqp-amqp-user}{amqp-user} \DTcomment{[Optional]}.
    .5 \intlink{sec:yaml-generic-ssl-verify}{ssl-verify} \DTcomment{[Optional] Default: True}.
    .4 \intlink{sec:yaml-resolver-allowed-agent-tags}{allowed-agent-tags} \DTcomment{[Required]}.
    .4 \intlink{sec:yaml-resolver-capture-resolver-logs}{capture-resolver-logs} \DTcomment{[Optional] Default: False}.
    .4 \intlink{sec:yaml-resolver-default-agent-tag}{default-agent-tag} \DTcomment{[Optional]}.
    .4 \intlink{sec:yaml-resolver-private-key}{private-key} \DTcomment{[Required]}.
    .4 \intlink{sec:yaml-resolver-resolver-tag-key}{resolver-tag-key} \DTcomment{[Optional] Default: resolver}.
    .4 \intlink{sec:resolver-scan-retries}{scan-retries}\DTcomment{[Optional] Default: 3}.
    .4 \intlink{sec:resolver-scan-timeout-seconds}{scan-timeout-seconds}\DTcomment{[Optional] Default: 10800}.
}


\pagebreak
\paragraph{YAML \texttt{scan-config} Elements}\label{sec:scan-config-elements}
\noindent\\

\dirtree{%
    .1 <root>.
    .2 \intlink{sec:yaml-scm-monikers}{<scm moniker>} \DTcomment{[Required: \textbf{bbdc}, \textbf{adoe}, \textbf{gh}, \textbf{gl}]}.
    .3 \intlink{sec:yaml-moniker-scan-config}{scan-config} \DTcomment{[Optional]}.
    .4 \intlink{sec:yaml-scan-config-default-scan-engines}{default-scan-engines} \DTcomment{[Optional]}.
    .4 \intlink{sec:yaml-scan-config-default-project-tags}{default-project-tags} \DTcomment{[Optional]}.
    .4 \intlink{sec:yaml-scan-config-default-scan-tags}{default-scan-tags} \DTcomment{[Optional]}.
}


\input{operation/yaml/root.tex}
\subsubsection{YAML Element: <scm moniker>.connection}\label{sec:yaml-moniker-connection}
A block element where the child elements define the SCM connection parameters for this service definition.

\subsubsection{YAML Element: <scm moniker>.cxone}\label{sec:yaml-moniker-cxone}
A block element where the child elements define the connection configuration for the \cxone API. 

\subsubsection{YAML Element: <scm moniker>.feedback}\label{sec:yaml-moniker-feedback}
A block element where the child elements define the configuration for feedback workflows. 

\subsubsection{YAML Element: <scm moniker>.kickoff}\label{sec:yaml-moniker-kickoff}
A block element where the child elements define the configuration for kickoff workflows. 

\subsubsection{YAML Element: <scm moniker>.project-groups}\label{sec:yaml-moniker-project-groups}
A block element where the child elements define the configuration for automatic project group assignment. 

\subsubsection{YAML Element: <scm moniker>.project-naming}\label{sec:yaml-moniker-project-naming}
A block element where the child elements define the configuration for dynamic project naming. 

\subsubsection{YAML Element: <scm moniker>.scan-agent}\label{sec:yaml-moniker-scan-agent}
A block element where the child elements define the configuration for support of scan agents. 

\subsubsection{YAML Element: <scm moniker>.repo-match}\label{sec:yaml-moniker-repo-match}
A regex applied to the source repository.  If the webhook payload has
a clone URL that matches the regex, this service definition is used to orchestrate scanning
for the received event.  As of \cxoneflow 2.1.0, this element will no longer allow regular
expressions such as \texttt{.*} that match any arbitrary value.  It is suggested to use
a regular expression that uses the start position anchor (\^{}) along with an expression
that matches the URL of the SCM that will be expected to emit the event.  The ending of
the expression can be followed by \texttt{.*} to match events from any repository.

This has been changed in consideration of secure deployment practices.  Please refer to
Appendix \ref{sec:cxoneflow-security} for more information.

\subsubsection{YAML Element: <scm moniker>.scan-config}\label{sec:yaml-moniker-scan-config}
A block element where the child elements define the default scan configuration for this service endpoint.

\subsubsection{YAML Element: <scm moniker>.service-name}\label{sec:yaml-moniker-service-name}
A moniker for the service definition. The moniker is used for logging and workflow purposes.








\input{operation/yaml/scan-config.tex}
\input{operation/yaml/cxone.tex}
\chapter{Feedback Workflows}\label{sec:feedback-workflows}


\section{Overview}

When webhook events are received by \cxoneflowns, the content of the event
payload is evaluated to determine if a scan should be started.  If a scan
is started, a background workflow executes that monitors the scan progress.
When the scan is completed and produces results, the end of the workflow
will transform the results for the purpose of presenting them to the user
for evaluation.

This section describes the workflows as implemented by \cxoneflowns.  Integration
with the workflows to perform parallel or replacement activities is possible
by utilizing the internal workflow messaging. Feedback output can be turned off 
via configuration to execute the messaging workflows without writing feedback; this
allows integration scenarios that replace default feedback output.  If the
feedback output remains enabled via configuration, this allows integration
scenarios for customized workflows to execute in parallel with feedback
output implemented in \cxoneflowns. Please see Appendix 
\ref{sec:amqp-workflow-orch} for details related to workflow integration.




\section{Scan Monitoring}

A scan in \cxone can be performed using one or more different scan engines.
A scan that has been completed will generally decide the next step in the workflow.
Keep in mind that a "completed" scan may mean the scan ended with any of the
following outcomes:

\begin{itemize}
    \item Scans for all requested engines completed successfully, each having zero
    or more results.
    \item The scan may have failed with no engines ever having started a scan.
    \item The scan may have partially failed where one or more engines successfully
    completed a scan and one or more engines had a scan failure.
    \item The scan may have been cancelled before any engine produced results.
    \item The scan may have been cancelled after one or more engines produced
    results but before all the engines were able to produce results.
\end{itemize}

When the scan is completed (which doesn't imply that it was successful), a message
is enqueued that starts the next step in the workflow.  
Figure \ref{fig:polling-flowchart} shows the scan polling algorithm that is followed
to determine when to enqueue a message that starts the next step in the workflow.

\begin{figure}[ht]
    \includegraphics[scale=.75]{graphics/cxoneflow-diagrams-Polling Algorithm.png}
    \centering
    \caption{Scan Polling Algorithm}
    \label{fig:polling-flowchart}
\end{figure}

\section{Annotation Workflow}\label{sec:annotation-workflow}

The annotation workflow is intended to perform any type of operations that
would inform a user of the following scan dispositions:

\begin{itemize}
    \item Started
    \item Cancelled
    \item Failed
\end{itemize}

\noindent\\The messaging workflow is currently very simple and follows the algorithm
shown in Figure \ref{fig:recovery-flowchart}.

\section{Pull Request Feedback Workflow}\label{sec:pull-request-workflow}

The pull request feedback workflow is invoked when a scan completes with the
following conditions:

\begin{itemize}
    \item A scan is fully completed on all requested scan engines.
    \item A scan is in the \texttt{Partial} result state with no
    engines currently running a scan.
\end{itemize}

The feedback workflow messaging follows the algorithm shown in
Figure \ref{fig:recovery-flowchart}.  The feedback that is emitted
for pull-request feedback is a summary of components found in the
\extlink{https://docs.checkmarx.com/en/34965-182434-checkmarx-one-reporting.html}{Improved Scan Report}
as generated by the
\extlink{https://checkmarx.stoplight.io/docs/checkmarx-one-api-reference-guide/branches/main/7bf86350cfe72-create-a-report}{"create a report"}
\cxone API.  Some results can be excluded via configuration,
as described in Section \ref{sec:yaml-config}.  If the results of a particular
engine are not included in the Improved Scan Report, they are not available for
publication in the feedback comment.

\subsection{Pull Request Comment Contents}

Each source control system will impose a maximum size for content written in
a pull request comment.  Since scans can produce an unpredictable number of
results for each engine scanned, it is possible that a full itemized
summary of results in a pull request comment will exceed the maximum comment
size.  \cxoneflow will write the full itemized summary of results if the
content size is less than the maximum comment content size.  In the event that
the maximum comment content size would be exceeded, a simple count of
vulnerabilities by severity and engine is written as the comment content.


\subsubsection{Header}

An example header of the pull-request comment is shown in Figure
\ref{fig:pr-header-section}.  It contains a link to the scan
and an indicator of the scan status of the selected engines.  An indicator
of a red "X" next to an engine indicates the scan status of anything other
than successful completion of the scan running on that engine.

\begin{figure}[ht]
    \includegraphics[width=\textwidth]{graphics/pr-header.png}
    \caption{Pull-Request Comment Header}
    \label{fig:pr-header-section}
\end{figure}

\subsubsection{Summary of Vulnerabilities}

An example summary of vulnerability counts by severity and engine 
is shown in Figure \ref{fig:pr-summary}.  The calculated counts will not
include vulnerabilities marked as "Not Exploitable".  Severities that
are configured for exclusions, as documented in Section
\ref{sec:yaml-config}, are not included in the table.  When no 
vulnerabilities of a given severity with a status other than 
"Not Exploitable" are found, the count is indicated by "N/R" (none reported).


The counts for SCA scans only reflect the reported vulnerabilities in the
Risks tab of the SCA results viewer.  The count will differ from the summary
of SCA results shown when selecting a scan in Scan History; the scan
history view shows a combined count of all categories of SCA results.

\begin{figure}[ht]
    \includegraphics[width=\textwidth]{graphics/pr-summary.png}
    \caption{Pull-Request Summary Count by Severity and Engine}
    \label{fig:pr-summary}
\end{figure}


\subsubsection{SAST Results}

An example of the SAST result summary written to the pull-request comment
is shown in Figure
\ref{fig:pr-sast-section}.  The name of the issue is a link to a detailed
explanation of the vulnerability that includes remediation advice.  A
link to the line of the source code leads to the source file located in the
repository.  The Checkmarx Insight link opens the triage view for the vulnerable
data flow.

\begin{figure}[ht]
    \includegraphics[width=\textwidth]{graphics/pr-sast.png}
    \caption{Pull-Request Comment SAST Results}
    \label{fig:pr-sast-section}
\end{figure}

\subsubsection{SCA Results}

An example of the SCA result summary written to the pull-request comment
is shown in Figure
\ref{fig:pr-sca-section}.  The Checkmarx Insight link opens the triage view 
for the vulnerable package. 

\begin{figure}[ht]
    \includegraphics[width=\textwidth]{graphics/pr-sca.png}
    \caption{Pull-Request Comment SCA Results}
    \label{fig:pr-sca-section}
\end{figure}

\subsubsection{IAC Results}

An example of the IAC result summary written to the pull-request comment
is shown in Figure
\ref{fig:pr-iac-section}. A
link to the line of the source code leads to the source file located in the
repository.  The Checkmarx Insight link opens the triage view for the vulnerable
configuration.

\begin{figure}[ht]
    \includegraphics[width=\textwidth]{graphics/pr-iac.png}
    \caption{Pull-Request Comment IaC Results}
    \label{fig:pr-iac-section}
\end{figure}

\subsubsection{Resolved Issues}

An example of the Resolved Issues summary written to the pull-request comment
is shown in Figure
\ref{fig:pr-resolved-section}. This section appears only if a scan results in
some of the issues are found to have been resolved by a new scan.
The Checkmarx Insight link opens the triage view for the vulnerable
data flow.

\begin{figure}[ht]
    \includegraphics[width=\textwidth]{graphics/pr-resolved.png}
    \caption{Pull-Request Comment Resolved Section}
    \label{fig:pr-resolved-section}
\end{figure}


\input{operation/yaml/kickoff.tex}
\input{operation/yaml/project-groups.tex}
\input{operation/yaml/project-naming.tex}
\input{operation/yaml/scan-monitor.tex}
\input{operation/yaml/exclusions.tex}
\input{operation/yaml/connection.tex}
\input{operation/yaml/api-auth.tex}
\input{operation/yaml/clone-auth.tex}
\subsubsection{YAML Element: <scm moniker>.scan-agent.allowed-agent-tags}\label{sec:yaml-scan-agent-allowed-agent-tags}
A list of scan agent tags that are handled by the \cxoneflow service.  The agent tags come from
\cxone project tags using the \texttt{<resolver-tag-key>:<agent tag>} format.  

\subsubsection{YAML Element: <scm moniker>.scan-agent.capture-resolver-logs}\label{sec:yaml-scan-agent-capture-resolver-logs}
If set to True, any logs emitted by the scan agent are transported back to the \cxoneflow server and
emitted in the \cxoneflow logs.  This can be used to troubleshoot issues encountered by scan agents.

\subsubsection{YAML Element: <scm moniker>.scan-agent.default-agent-tag}\label{sec:yaml-scan-agent-default-agent-tag}
An agent tag found in the list of tags configured in \texttt{allowed-agent-tags}.  If provided,
any \cxone project that does not have an agent tag (assigned with the format \texttt{<resolver-tag-key>:<agent tag>})
will direct a scan to Scan Agents with this tag.

\subsubsection{YAML Element: <scm moniker>.scan-agent.private-key}\label{sec:yaml-scan-agent-private-key}
The value specifies a file name found under the path defined by \texttt{secret-root-path}
containing an unencrypted RSA or elliptic curve private key.  The associated public key will be distributed to
Scan Agents.

\subsubsection{YAML Element: <scm moniker>.scan-agent.resolver-tag-key}\label{sec:yaml-scan-agent-resolver-tag-key}
The \cxone project tag key value used to find the scan agent tag that should be used for a delegated scan.
This is described in more detail in Section \ref{sec:scan-agents}.

\subsubsection{<scm moniker>.scan-agent.scan-retries}\label{sec:scan-agent-scan-retries}
The number of times a request for a delegated scan will be retried after the initial request if no Scan Agent handles the scan.
Any scan request not handled by a Scan Agent before the \intlink{sec:scan-agent-scan-timeout-seconds}{scan-timeout-seconds} setting 
will be evaluated for resubmission.  If the number of retries exceeds this amount, the scan will be returned in a failed state.

\subsubsection{<scm moniker>.scan-agent.scan-timeout-seconds}\label{sec:scan-agent-scan-timeout-seconds}
The number of seconds before a scan request is not selected by a Scan Agent before it times out.  When the scan
request times out, it may be resubmitted or aborted as a failed scan.

\subsubsection{YAML Element: amqp}\label{sec:yaml-generic-amqp}
The connection parameters for an AMQP endpoint used for workflow orchestration and scan agent coordination. 
If not set, the internal RabbitMQ instance in the \cxoneflow container is used by default.

\subsubsection{YAML Element: amqp-url}\label{sec:yaml-generic-amqp-amqp-url}
This can be one of the following values:

\begin{itemize}
  \item The AMQP/AMQPS URL for the AMQP endpoint.
  \item The name of a file container a secret located at the path defined by \texttt{secret-root-path}.
\end{itemize}


\subsubsection{YAML Element: amqp-user}\label{sec:yaml-generic-amqp-amqp-user}
If the user name is not included in the AMQP URL, the provided value corresponds to a file name found under
the path defined by \texttt{secret-root-path}.

\subsubsection{YAML Element: amqp-password}\label{sec:yaml-generic-amqp-amqp-password}
If the password is not included in the AMQP URL, the provided value corresponds to a file name found
under the path defined by \texttt{secret-root-path}.


\subsubsection{YAML Element: ssl-verify}\label{sec:yaml-generic-ssl-verify}
See discussion in Section \ref{sec:ssl-verify-general}.  Defaults to True using the OS trusted CA certificates.

\subsubsection{YAML Element: timeout-seconds}\label{sec:yaml-generic-timeout-seconds}
The number of seconds before a request for API results times out.

\subsubsection{YAML Element: retries}\label{sec:yaml-generic-retries}
The number of retries when the request fails.

\subsubsection{YAML Element: retry-delay}\label{sec:yaml-generic-retry-delay}
The maximum number of seconds to wait between retries of failed requests.

\subsubsection{YAML Element: proxies}\label{sec:yaml-generic-proxies}
A YAML dictionary of \texttt{<scheme>:<url>} pairs to use a proxy server for requests. 
For a format of key/value pairs, see: \extlink{https://requests.readthedocs.io/en/latest/user/advanced/\#proxies}{Python "requests" proxies}.











\chapter{\cxoneflowtext\space Security Considerations}\label{sec:cxoneflow-security}


\section{SCM Shared Secret}

The SCM event configurations all utilize a "shared secret" token that is used to validate that
a webhook event is emitted from a trusted source SCM.  In some SCMs, the token is used to
demonstrate mutual knowledge if the value using HMAC signing of the payload.  This prevents
the need to expose the shared secret.  While not foolproof, a sufficiently complex shared
secret will be difficult to reverse and thus prevent exposure to a certain extent.

In most of the SCMs, however, the shared secret value is sent in either plaintext or as base-64 
encoded text.  Anyone monitoring the communication between the SCM and \cxoneflow can obtain
the shared secret value when it is sent as part of the event payload.  Using SSL to encrypt
communication between the SCM and \cxoneflow can also minimize the risk of exposure
of the shared secret to a certain extent.

Validation of received webhook event payloads using the shared secret is often the only mechanism
available to prevent someone from sending a false webhook event payload.
Unfortunately, the shared secret usually needs to be shared with those that are configuring
SCM webhooks to send events to \cxoneflowns.  As the number of people that know the shared secret
grows, the risk of it being exposed to a threat actor becomes greater.  Anyone that knows
the shared secret and can make a network connection to the \cxoneflow endpoint has the ability
to send a false webhook event payload that could cause undesirable consequences.

The \cxoneflow configuration uses a regular expression to match the service definition
that is intended to handle a received webhook event.  The \texttt{repo-match} element
configured with an all-matching regular expression such as \texttt{.*} means that the
service definition will handle an event for any repository clone URL presented in the
received webhook payload.  If a threat actor with the shared secret were to send a modified payload
containing a clone URL of their choosing, it would be accepted for handling by
the matching service definition.  This would potentially allow exposure of the
clone credentials via a SSRF attack.

The following recommendations can be used to mitigate and possibly prevent any
issues that may occur due to exposure of the shared secret.

\subsection{Mitigating Shared Secret Exposure}


\subsubsection{Avoid All-Matching Regular Expressions}

To prevent an SSRF scenario due to an all-matching regular expression, 
\cxoneflow as of version 2.1.0 will not start if a configured \texttt{repo-match} element
will match arbitrary values. Using a regular expression that matches the transport and FQDN
of clone URLs expected in each event payload will prevent falsified events from being matched to a 
service definition.  As an example, most regular expressions would be in a format similar to the one below:

\begin{code}{General Clone URL Regular Expression}{}{}
^http(s)?:(\/){2}my\.scm\.corp\.com.*
\end{code}

In the case of cloud-hosted Azure DevOps, an slightly different regular expression
is needed.  The reason for this is that pull request events emitted from the multi-tenant ADO cloud have
the organization name in the clone URL.  Below is an example of an Azure DevOps cloud
service endpoint:

\begin{code}{Minimal YAML Configuration}{Azure DevOps Cloud}{}
  secret-root-path: /run/secrets
  server-base-url: https://cxoneflow.mydomain.com:8443/
  
  adoe:
      - service-name: MyADO
        repo-match: ^http(s)?:(\/){2}(.+@)?dev\.azure\.com.*
        connection:
        base-url: https://scm.corp.com/
        shared-secret: scm-shared-secret
        api-auth:
          username: scm-username-secret
          password: scm-password-secret
        cxone:
          tenant: mytenant
          oauth:
            client-id: my-cxone-client-id
            client-secret: my-cxone-client-secret
          iam-endpoint: US
          api-endpoint: US
\end{code}
  
\subsubsection{Limit Outbound Network Connectivity of the \cxoneflowtext\space Endpoint}

When \cxoneflow is operating, there are only very few connections to external machines
needed.  These are generally:

\begin{itemize}
  \item The IAM and API URLs hosting the \cxone tenant.
  \item The SCM endpoints for API access and cloning.
  \item If an external AMQP server is used, the external AMQP endpoint.
\end{itemize}

Egress from the \cxoneflow instance subnet can be controlled by firewall rules or
through an outbound proxy.  If using a proxy, the \intlink{sec:yaml-generic-proxies}{proxies}
YAML configuration element can be used to provide the proxy server's connection information.

\subsubsection{Limit Inbound Network Connectivity of the \cxoneflowtext\space Endpoint}

The only required inbound connections to \cxoneflow are from SCMs delivering webhook event
payloads to the \cxoneflow endpoint.  Controlling the source of connections made to \cxoneflow using firewall rules
or reverse-proxy filtering can limit the risk that would come with an exposed shared secret.

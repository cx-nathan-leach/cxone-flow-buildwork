
\section{Pull Request Feedback Workflow}\label{sec:pull-request-workflow}

The pull request feedback workflow is invoked when a scan completes with the
following conditions:

\begin{itemize}
    \item A scan is fully completed on all requested scan engines.
    \item A scan is in the \texttt{Partial} result state with no
    engines currently running a scan.
\end{itemize}

The feedback workflow messaging follows the algorithm shown in
Figure \ref{fig:recovery-flowchart}.  The feedback that is emitted
for pull-request feedback is a summary of components found in the
\extlink{https://docs.checkmarx.com/en/34965-182434-checkmarx-one-reporting.html}{Improved Scan Report}
as generated by the
\extlink{https://checkmarx.stoplight.io/docs/checkmarx-one-api-reference-guide/branches/main/7bf86350cfe72-create-a-report}{"create a report"}
\cxone API.  Some results can be excluded via configuration,
as described in Section \ref{sec:yaml-config}.  If the results of a particular
engine are not included in the Improved Scan Report, they are not available for
publication in the feedback comment.

\subsection{Pull Request Comment Contents}

Each source control system will impose a maximum size for content written in
a pull request comment.  Since scans can produce an unpredictable number of
results for each engine scanned, it is possible that a full itemized
summary of results in a pull request comment will exceed the maximum comment
size.  \cxoneflow will write the full itemized summary of results if the
content size is less than the maximum comment content size.  In the event that
the maximum comment content size would be exceeded, a simple count of
vulnerabilities by severity and engine is written as the comment content.


\subsubsection{Header}

An example header of the pull-request comment is shown in Figure
\ref{fig:pr-header-section}.  It contains a link to the scan
and an indicator of the scan status of the selected engines.  An indicator
of a red "X" next to an engine indicates the scan status of anything other
than successful completion of the scan running on that engine.

\begin{figure}[ht]
    \includegraphics[width=\textwidth]{graphics/pr-header.png}
    \caption{Pull-Request Comment Header}
    \label{fig:pr-header-section}
\end{figure}

\subsubsection{Summary of Vulnerabilities}

An example summary of vulnerability counts by severity and engine 
is shown in Figure \ref{fig:pr-summary}.  The calculated counts will not
include vulnerabilities marked as "Not Exploitable".  Severities that
are configured for exclusions, as documented in Section
\ref{sec:yaml-config}, are not included in the table.  When no 
vulnerabilities of a given severity with a status other than 
"Not Exploitable" are found, the count is indicated by "N/R" (none reported).


The counts for SCA scans only reflect the reported vulnerabilities in the
Risks tab of the SCA results viewer.  The count will differ from the summary
of SCA results shown when selecting a scan in Scan History; the scan
history view shows a combined count of all categories of SCA results.

\begin{figure}[ht]
    \includegraphics[width=\textwidth]{graphics/pr-summary.png}
    \caption{Pull-Request Summary Count by Severity and Engine}
    \label{fig:pr-summary}
\end{figure}


\subsubsection{SAST Results}

An example of the SAST result summary written to the pull-request comment
is shown in Figure
\ref{fig:pr-sast-section}.  The name of the issue is a link to a detailed
explanation of the vulnerability that includes remediation advice.  A
link to the line of the source code leads to the source file located in the
repository.  The Checkmarx Insight link opens the triage view for the vulnerable
data flow.

\begin{figure}[ht]
    \includegraphics[width=\textwidth]{graphics/pr-sast.png}
    \caption{Pull-Request Comment SAST Results}
    \label{fig:pr-sast-section}
\end{figure}

\subsubsection{SCA Results}

An example of the SCA result summary written to the pull-request comment
is shown in Figure
\ref{fig:pr-sca-section}.  The Checkmarx Insight link opens the triage view 
for the vulnerable package. 

\begin{figure}[ht]
    \includegraphics[width=\textwidth]{graphics/pr-sca.png}
    \caption{Pull-Request Comment SCA Results}
    \label{fig:pr-sca-section}
\end{figure}

\subsubsection{IAC Results}

An example of the IAC result summary written to the pull-request comment
is shown in Figure
\ref{fig:pr-iac-section}. A
link to the line of the source code leads to the source file located in the
repository.  The Checkmarx Insight link opens the triage view for the vulnerable
configuration.

\begin{figure}[ht]
    \includegraphics[width=\textwidth]{graphics/pr-iac.png}
    \caption{Pull-Request Comment IaC Results}
    \label{fig:pr-iac-section}
\end{figure}

\subsubsection{Resolved Issues}

An example of the Resolved Issues summary written to the pull-request comment
is shown in Figure
\ref{fig:pr-resolved-section}. This section appears only if a scan results in
some of the issues are found to have been resolved by a new scan.
The Checkmarx Insight link opens the triage view for the vulnerable
data flow.

\begin{figure}[ht]
    \includegraphics[width=\textwidth]{graphics/pr-resolved.png}
    \caption{Pull-Request Comment Resolved Section}
    \label{fig:pr-resolved-section}
\end{figure}


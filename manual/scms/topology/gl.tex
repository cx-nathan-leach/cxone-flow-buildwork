Each Gitlab self-hosted instance has a system-wide Service Hook configuration where webhook
configurations for \cxoneflow can be deployed.  When webhooks are deployed at this scope, 
all repositories regardless of location in the Gitlab topology will send webhook events to \cxoneflowns.
The Service Hook capability is available for any Gitlab subscription level.

Configuration of webhooks using the system-wide scope is generally the preferred
method of webhook deployment for Gitlab.  This is only available for Gitlab self-hosted instances.

All variations of Gitlab have zero or more \textbf{Group} and/or \textbf{Repository} logical units
at the root of the organizational topology.  Each \textbf{Group} logical unit can also hold zero or more
\textbf{Group} and/or \textbf{Repository} units.  Webhook configuration at the \textbf{Group}
scope in groups closest to the root of the topology are generally the preferred
method of webhook deployment for Gitlab.  Deployment of webhook configurations at lower
group scopes should be avoided.

Webhooks can be deployed at the scope of each \textbf{Repository} if desired.  The number of repositories
in a large enterprise generally makes deployment at the \textbf{Repository} scope useful
only for testing purposes.


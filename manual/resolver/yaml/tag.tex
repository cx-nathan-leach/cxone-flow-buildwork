\subsubsection{<agent tag>.disable-resolver}\label{sec:agent-disable-resolver}
Defaults to \texttt{False}.  Set to \texttt{True} to disable execution of \scaresolver.  This is typically used with 
\intlink{sec:agent-pre-scan}{pre-scan} configured to execute a scan step that does not require the use of \scaresolver.

If this is set to \texttt{True}, options \intlink{sec:agent-resolver-path}{resolver-path} and 
\intlink{sec:agent-run-with-container}{run-with-container} are ignored.


\subsubsection{<agent tag>.pre-scan}\label{sec:agent-pre-scan}
This element contains elements to support \intlink{sec:scan-agent-prescan}{pre-scan scripting}.

\subsubsection{<agent tag>.pre-scan.container-image-tag}\label{sec:agent-pre-scan-container-image-tag}
This is a tag to a container image where the configured \intlink{sec:scan-agent-prescan}{script} will execute.

\subsubsection{<agent tag>.pre-scan.resolver-before-script}\label{sec:agent-pre-scan-resolver-before-script}
If set to \texttt{False} (the default), \scaresolver executes after the \intlink{sec:agent-pre-scan-script}{pre-scan script}.
If set to \texttt{True}, \scaresolver executes before the \intlink{sec:agent-pre-scan-script}{pre-scan script}. If \scaresolver
execution has been disabled with the \intlink{sec:agent-disable-resolver}{disable-resolver} option, this setting is ignored.


\subsubsection{<agent tag>.pre-scan.run-as-agent}\label{sec:agent-pre-scan-run-as-agent}
The default setting is \texttt{True}.  If set to \texttt{True}, the container runs with the UID:GID of
the scan agent.  This is to allow for the overlay file system to read/write to the mapped code directory
while executing the pre-scan script.  If set to \texttt{False}, the container runs as the container default
user.

When running as the container default user, anything written to the mapped code directory may prevent the scan
agent from having read/write access after the container exits.  The pre-scan script should set the read/write
permissions on or remove files written during the container execution prior to exiting the container.  This will
allow the scan agent to remove the temporary copy of the code when the pre-scan execution is complete.


\subsubsection{<agent tag>.pre-scan.script}\label{sec:agent-pre-scan-script}
The script to execute in the running container image.  This can be a single line or a multi-line
string using the \extlink{https://yaml.org/spec/1.2.2/\#812-literal-style}{YAML literal scalar style} syntax
by using the bar character ("\texttt{|}") as the value and indenting the next several lines to form the
complete script.


\subsubsection{<agent tag>.pre-scan.shell}\label{sec:agent-pre-scan-shell}
By default, the pre-scan script will execute in the container's shell located at \texttt{/bin/sh}.  This is a standard shell location
in most Linux distributions.  The path to an alternate shell on the container image can optionally be defined here.

\subsubsection{<agent tag>.public-key}\label{sec:agent-public-key}
The value specifies a file name found under the path defined by \texttt{secret-root-path} containing a 
public key that matches the server's configured \intlink{sec:yaml-resolver-private-key}{\texttt{private-key}} setting.


\subsubsection{<agent tag>.resolver-opts}\label{sec:agent-resolver-opts}
This is a dictionary of
\extlink{https://docs.checkmarx.com/en/34965-132888-checkmarx-sca-resolver-configuration-arguments.html\#UUID-bc93274b-c1c7-ea47-9556-3bd8900711dc_id_CheckmarxSCAResolverConfigurationArguments-ConfigurationArguments-TablesandSamples}{configuration arguments}
passed to \scaresolver when executing.  The options are used to provide static values for resolver execution configuration.  Some of the
options may clash with execution options provided by the agent; options that would clash with how the agent executes \scaresolver are ignored.  

The \texttt{resolver-opts} section is a dictionary of key and key/value pairs that correspond to
\extlink{https://docs.checkmarx.com/en/34965-132888-checkmarx-sca-resolver-configuration-arguments.html}{command line options} for \scaresolver.
The options that can be used are limited considering some of the options are used by the \cxoneflow integration to execute \scaresolver.

These options, if set, will be ignored:

\begin{itemize}
  \item logs-path
  \item a | account
  \item containers-result-path
  \item resolver-result-path
  \item project-name
  \item authentication-server-url
  \item p | password
  \item sso-provider
  \item sca-app-url
  \item s | scan-path
  \item server-url
  \item u | username
  \item project-tags
  \item scan-tags
  \item bypass-exitcode
  \item no-upload-manifest
  \item help
  \item manifests-path
  \item t | project-teams
  \item q | quiet
  \item save-evidence-path
  \item severity-threshold
  \item report-content
  \item report-extension
  \item report-path
  \item report-type
  \item sast-result-path
  \item cxpassword
  \item cxuser
  \item cxprojectid
  \item cxprojectname
  \item cxserver
\end{itemize}

\subsubsection{<agent tag>.resolver-path}\label{sec:agent-resolver-path}
The path to the \scaresolver executable that has been installed on the system running the Scan Agent.


\subsubsection{<agent tag>.resolver-run-as}\label{sec:agent-resolver-run-as}
The name of a user account that will run the \scaresolver when executed in a shell (but not as a container).  This
is an advanced configuration that will require additional configuration for your platform.  

If not provided, the \scaresolver is executed as the same user that is running the Scan Agent service.

\subsubsection{<agent tag>.scan-agent-work-path}\label{sec:scan-agent-work-path}
A path where temporary files are written during the operation of the Scan Agent.  This also serves as the home
directory for the Scan Agent user and the user defined in \texttt{resolver-run-as}.

\subsubsection{<agent tag>.run-with-container}\label{sec:agent-run-with-container}
A YAML dictionary with key/value pairs used to define running \scaresolver in a container.  If this is supplied,
the configured \texttt{resolver-path} is ignored and \scaresolver will not be invoked in a shell.  The dependency
tree collected by \scaresolver will be done by executing build tools defined in the container.  This is useful for
organizations that utilize containerized build environments in their CI/CD pipeline build scripts.

The use of containers to run \scaresolver is not supported on Windows platforms.

\subsubsection{<agent tag>.run-with-container.container-image-tag}\label{sec:agent-container-image-tag}
The container tag that is found in one of the logged-in container registries.  This container tag is used by the \toolkit
to create an extended image with \scaresolver installed.

\subsubsection{<agent tag>.run-with-container.supply-chain-toolkit-path}\label{sec:agent-supply-chain-toolkit-path}
The path where the \toolkit build environment is installed.

\subsubsection{<agent tag>.run-with-container.use-running-gid and\\<agent tag>.run-with-container.use-running-uid}\label{sec:agent-use-running}
These options are True by default.  This causes the image built by the \toolkit to use the
UID and primary GID of the user running the Scan Agent when defining a non-root
user in the extended image.  

The reason for this is that when \scaresolver is executed in the container, temporary paths
in the \texttt{resolver-work-path} are mapped to the container.  Files created by the container
will have the UID/GID of the running container's user when created.  Since the UID/GID of the
container matches the UID/GID of the Scan Agent, the files that remain after
the container exits can be controlled by the Scan Agent.

Setting these values to False should only be done in circumstances where the UID/GID for written files
should be defined by the container.  This scenario may never practically exist.

